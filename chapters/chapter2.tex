%!TEX root = ../username.tex
\chapter{Game-Based Learning}
Game-based learning is a subset of serious games. Serious games are:
	\begin{quote}
		"played with a computer in accordance with specific rules and produced, marketed, or used for purposes other than pure entertainment; these include, but are not limited to, educational computer games (edutainment) and advertainment" \cite{Bontchev2015}.
	\end{quote}
 Advertainment being any kind of entertainment made with the goal of advertisement. Game-based learning uses entertainment to capture peoples' attention teaching them in a fun and engaging way. There are many advantages to using game-based learning such as increasing the users motor and spatial skills, training, and treating mental illness. 

\section{Applications}  
There are many applications of serious games and game-based learning. Serious games are used in the military and in the government for training. Serious games can be used to train personal by simulating a multitude of environments. Thus providing exposure before entering a new situation. Simulations allow the military and government to train effectively with a significant decrease in cost compared to traditional methods. Serious games can also be created for educational purposes; the goal of these educational games is to train the player in a discipline. Surgeons, for example, can use a game to practice a risky surgery for a specific patient before the surgery is actually performed. Games made to promote health care may include games like "Dance Dance Revolution", where this kind of physical fitness game is referred to as "exergaming". Serious games can be used to diagnose and treat mental illnesses, for example, games can be used in distraction therapy and exposure therapy \cite{Susi2007}. Distraction therapy is a way of helping someone cope with a difficult situation by keeping their attention elsewhere. Exposure therapy is a psychological treatment that helps people confront their fears. This is done by gradually introducing the patient to their fears in a safe environment \cite{Susi2007}. Games can also be used to help people recover and rehabilitate from accidents. Cognitive functioning can also be increased with the use of games. This includes activities like memory training and development of analytical and strategic skill \cite{Susi2007}. 
\begin{figure}[!ht]
	\begin{center}
		\woopic{3edGradeGame}{.6}
	\end{center}
	\caption{A screen shot of the Trials of the Acropolis \cite{Chintiadis}} \label{fig:3edGradeGame}
\end{figure}Games can be used to teach students in subjects such as spelling, reading, and history. For example, one game, titled Trials of the Acropolis, can be used to teach about history using the virtual reality platform. It's goal is to expose students, specifically third graders, to ancient Greek mythology. The game consists of many levels, each of which focus on a different myths from mythology such as; Hercules, Odyssey, and Gods and Titans. A screen shot of the game is shown in Figure~\ref{fig:3edGradeGame}. Each level of the game consists of many scenarios that involve the player solving puzzles and taking quizzes. These puzzles and quizzes help the student test and use what they have learned from the specific level of the game \cite{Chintiadis}. 


\section{Why it Works}
There are four main reasons why game-based learning is effective. The first is the amount of time the learner spends on the material. It has been shown that students tend to spend more time with the assigned material when it is in the form of a game rather than when it is provided in other traditional representations. Second, specific neural receptors in the brain are developed while playing games. These newly developed receptors allow for the student to have better retention of the material. Third, students become more aware of differences in their surrounding stimuli. This allows them to recognize more subtle differences within their environment. Lastly, students tend to be capable of reacting to multiple stimuli at once \cite{LaValle2017}. 


A student spends more time with the assigned material while playing an educational game because they are, ideally, interested and engaged in it. This engagement allows the player to gain a deeper understanding of the material and achieve more meaningful learning and retention. This engagement can also lead to more interaction between the student and instructor, which can result in a higher understanding. There are many arguments as to why game-based learning works. Edgar Dale's Cone of Experience, shown in  Figure ~\ref{fig:whatpeoplerememerfigure}, can be used to explain the effectiveness of game-based learning \cite{Davis2015}.
\begin{figure}[!ht]
	\begin{center}
		\woopic{whatpeoplerememerfigure}{.6}
	\end{center}
	\caption{Edgar Dale's Cone of Experience \cite{Davis2015}} \label{fig:whatpeoplerememerfigure}
\end{figure}The Cone of Experience shows that people are more likely to remember 90\% of what they do, as opposed to 10\% or 20\% of what they read or hear, respectively. This means that when a person experiences something, they are more likely to retain information from that experience. Games, especially games in VR, allow people to have experiences they would otherwise not have. Section~\ref{guidlines} goes into more detail on how this experience can be created and be made effective.   


There have been many studies that test if using games, along with traditional teaching methods, actually improve student's scores. One study examines two groups of students enrolled in an engineering class called 'Materials and Methods of Construction' \cite{Tobias2014}. Both groups take the same exam but are given different ways to study the material. The control group is given the assigned reading while the experimental group is given the assigned reading and instructed to play SimCity 2000, a city construction game. Both groups were given a 20 question multiple choice and true/false exam. After the exam they were given a follow up survey. The experimental group out performed the control group in the exam and based on the follow up survey they enjoyed the computer simulation more than the reading. This shows that between a group of students who are given a game to learn the material versus other methods, the students who use a game tend to do better and have a more enjoyable experience. This may be due to the fact that they spend more time with the material because they enjoy the game or because of the material in the game itself. Either way, the students who used the game outperformed the students who did not \cite{Tobias2014}. 


In another study two kindergarten classes, children from ages 5-6, are analyzed. Most of these students are African-American, of low socioeconomic status, and have one parent. The experimental class, consists of 24 students while the control class consists of 23. The study is testing the students' test results in the subjects; spelling, math, and reading. The experimental group is given educational game consoles to use in class for 11 weeks. The control class did not receive the console but instead hold regular classes. Both the control and the experimental group improve in the 11 weeks however the experiential group shows significantly greater improvement in spelling and reading compared to the control group, but not in math \cite{Tobias2014}. This experiment shows how the use of games in classrooms can be beneficial to a students' learning. 

\section{Guidelines to Creating an Educational Game} \label{guidlines}
When creating an education game it is imperative to make the experience as realistic as possible. The goal of the game should be to activate the part of the player's brain that will be used to perform the learned activity \cite{Tobias2014}. In other words, the part of the brain that activates while playing the game should be the same part of the brain that activates while doing the intended activity. For example, if the game is teaching someone how to hammer a nail, while in the game a part of the players' brain should activate, then when that player goes to hammer a real nail the same part of the brain should activate. To create an experience that activates the correct parts of the brain, immersion is used. Immersion allows the player to be entirely focused and surrounded by the game, completely separating the player from the real world. There are many ways to create an immersive game, one way is to use realistic physics and human-like voices, instead of synthetic voices. Another way is to develop the game in a first-person perspective, allowing the player to experience the world through the eyes of the character. Including dialogue is another important aspect to include, making the player feel as if they are inside the game. The use of characters can be beneficial in creating immersion, these characters should be animated and able to interact with the player. The characters, or objects, can be used by the developers to help guide the players though the game without breaking the player's feeling of immersion. According to one study, these characters should not be aggressive but rather should act socially with the player as this is conducive to a good learning environment \cite{Tobias2014}.


Aesthetics are important in creating an immersive environment. One study asked experts in game design, computer science, and interactive media to fill out a survey. This survey consisted of questions regarding what game aesthetics they thought are most important for a students' perceived learning. Table~\ref{tbl:AestheticsPerceivedLearning} displays the results of experts reviews based on frequency of responses. 
\begin{table}[!ht]
	\begin{center}
		\caption{Game Aesthetics for Perceived Learning \cite{AbuBakar2017}\label{tbl:AestheticsPerceivedLearning}}
		\begin{tabular}{|l|c|c|c|}
			\hline 
			Game Aesthetic & Very Important & Somewhat Important & Not Important  \\ 
			\hline 
			Image & 6 & - & - \\ 
			\hline 
			Text & 5 & 1 & - \\ 
			\hline 
			Visual Perspective & 5 & 1 & - \\ 
			\hline 
			Color & 5 & 1 & - \\ 
			\hline 
			Graphic & 5 & 1 & - \\ 
			\hline 
			Layout & 5 & 1 & - \\ 
			\hline 
			Sound Effect & 3 & 3 & - \\ 
			\hline 
			Voice & 4 & 1 & 1 \\ 
			\hline 
			Music & 3 & 2 & 1 \\ 
			\hline 
			Shape & 3 & 2 & 1 \\ 
			\hline 
			Form & 3 & 2 & 1 \\ 
			\hline 
			Texture & 3 & 2 & 1 \\ 
			\hline 
		\end{tabular}
	\end{center}
\end{table}The experts showed that in their opinion images are most important for learning in a game. Thus images  should be a primary focus when developing an educational game \cite{AbuBakar2017}. 


The game should keep the player interested and motivated to continue to play \cite{Tobias2014}. Two ways of motivating the player to want to continue playing include intrinsic and extrinsic rewards. "Intrinsic motivation pushes us to act freely, on our own, for the sake of it; extrinsic motivation pulls us to act due to factors that are external to the activity itself, like reward or threat" \cite{Cataldo}. To create these rewards the game should include "a system of rewards and goals which motivate players, a narrative context which situates activity and establishes rules of engagement, learning content that is relevant to the narrative plot, and interactive cues that prompt learning and provide feedback" \cite{Cataldo}. Games that invoke deep learning and understanding rely on motivating the player to use higher order thinking rather than rote memorization or simple comprehension \cite{Cataldo}. 


Developers should be aware of the players cognitive load while in game and attempt to reduce it as much as possible. The cognitive load is the amount of effort someone must put into their working memory. Working memory is a part of short-term memory. By reducing the amount of effort required to think about the details of the game, the learner is able to place more energy in linking the new concepts to their prior knowledge. Figure~\ref{fig:MemoryandLearning} shows how information works it's way through the brain and into long term memory by way of the working memory. Reducing cognitive load can be accomplished by making the game flow naturally without the player having to struggle through it, making the game intuitive and realistic. 
\begin{figure}[!ht]
	\begin{center}
		\woopic{MemoryandLearning}{.7}
	\end{center}
	\caption{The process of information being processed and retained \cite{Tobias2014}} \label{fig:MemoryandLearning}
\end{figure} To help the player move through the game easily is it important to give them guidance. As previously stated, characters and objects can be used to provide the user with instruction and information. Pictures and demonstration, observing how to do an intended activity, are good methods of showing a player how to accomplish an intended activity. After the user is successful in completing the activity, it is important to give them positive feedback and encourage them to reflect on the correct response. The game should provide guidance that encourages the player to learn through discovery, allowing them to find the answer on their own rather than directly telling them what they are to do \cite{Tobias2014}. 


Immersion is not the only aspect that is important in creating a successful educational game. The game should be extremely engaging and allow the player to investigate the world. The player should be involved and invested in the game. This can be done by allowing the player to interact with the world (objects, characters, etc.), involve them in an interesting and immersive storyline, and empathetic characters \cite{Tobias2014}. More of how to do this is discussed in Chapter~\ref{gamedesign}.


While developing an education game it is important to work in teams as this allows for more perspectives and a more widely effective end result. Testing the game during development is integral in determining the effectiveness of the game in it's intended results. One way to do this is have users take a small test before playing the game. After they play the game for a set amount of time they should be tested again. If their test score had a significant change then the game was successful, if not then revisions to the game and the test should be done. After the user plays the game it is also important to ask the user their feelings on the game and what they felt was effective and what was not. It is also possible to have the user play the game while connected to a brain monitor allowing the experimenters to watch the players' brain activity. After recording their brain activity the experimenters can have the users perform the newly learned task with the game brain monitoring device. If in both the game and the activity, the same part of the brain was activated then the game was well designed. Tf not, the game should be revised and the test should be done again. It is also important to be aware of the emerging research findings. These tests should help the developers revise the game to make the game more effective, as stated earlier, the test and revise cycle should continue until the intended results are reached \cite{Tobias2014}.   

%flow psychology

\section{Conclusion}
Game-based learning can be a useful tool in teaching and learning. It allows a student to become more invested in the material and retain more ofsss information. It is possible to create a good game for education by making it an immerse and realistic experience.  
%stregthen this 