%!TEX root = ../username.tex
\chapter{Game Design} \label{gamedesign}
Many components are needed to create a game that is engaging and interesting. The game world should be consistent and give the player a sense of power. To create an interesting narrative, it is important for the story to have non-linear game play. This can be done by creating multiple storylines where the player's path is based on their actions and choices thus allowing for multiple outcomes. Another way to create non-linear game play is to allow multiple solutions to a singular problem. This allows players to find their own ways to complete a task and use their own creativity while doing so. This can also be done by giving the player the freedom of choosing the order in which tasks are completed. Lastly, a good game allows the the player to pick and choose which tasks they want to complete. Meaning all the game's tasks are not mandatory. This gives the player the ability to choose to only complete the tasks they find interesting while not having to be burdened with tasks they may find tedious or boring. By allowing the player to choose the tasks they complete the game will keep their interest longer \cite{Iii}. 


Structure can be given to the player through the use of rules and win states. While playing the game, users are encouraged to continue if they are given feedback. Feedback is like reinforcement, there are two kinds of reinforcement, positive and negative. Positive reinforcement is where the player is given a desired element. For example the acquiring of points, coins, ammo, or food would be positive reinforcement. Negative reinforcement is when an undesirable element is released from the player. For example, the loss of an ailment, an enemy, or a barrier would be considered negative reinforcement \cite{Skinner2014}. It is also important to allow the player to problem solve and find their own unique solutions. The game should spark creativity and invoke emotion in the player \cite{Nguyen2012}. 


A game should allow the player to have creative and destructive powers. Creative powers allow the player to bring something into existence, destructive powers allow the player to alter the environment. Giving a player these powers allow them to feel empowered and give them to have a sense of satisfaction and creativity \cite{Behavior2009}.  


An engaging game has challenges, some standard challenges include; time challenges, dexterity challenges, endurance challenges, memory challenges, logic challenges, and resource control challenges. These challenges allow for the player to become engaged and invested in the story. However, challenges should not be placed back to back with each other, allowing the player to recuperate from a challenge is important in keeping their endurance. The game's story must possess a clear beginning, middle, and finale, giving the player a sense of closure \cite{Nguyen2012}. 

\section{Storyboard}
One of the first processes in game design is creating a storyboard. A storyboard is a set of drawings that represent the different scenes in the game. The same for this project, it begins with the player entering an excavation site. Their supervisor explains how excavation is done and shows them the layout of the grounds. There is a tent where the player rests when they become fatigued. Once artifacts are found, they are transported by the jeep them to the lab where they are analyzed. After the team at the lab has analyzed the materials they update the computer records with the new  information about the artifact. A picture storyboard is located in the Appendix.                                     
%\begin{figure}
%	\centering
%	\includegraphics[width=0.7\linewidth]{../../StoryBoard/StoryBoard_Beginning}
%	\caption{}
%	\label{fig:storyboardbeginning}
%\end{figure}


\section{User Interface}
The user interface is an important aspect to any virtual reality game. Without an interface the user does not know what to do or how to do it. An interface allows the player to interact with the game. There are many kinds of user interfaces, each having their own advantages and disadvantages. Two commonly used interfaces are called diegetic and non-diegetic. A diegetic interface involves the interface as apart of the virtual world; a working clock for example. Non-diegetic interfaces involve information that is stationary. 2D image or text that sits on top of the screen; a health bar for example \cite{Iacovides2015}. Non-diegetic interfaces are rarely used in VR due to the users difficulty of focusing on an object that is constantly so close. Spacial interfaces are VR's version of non-diegetic interfaces and are discussed in further detail later in this section.   
\begin{figure}[!ht]
	\begin{minipage}[!ht]{6cm}
		\woopic{diagetic}{.9}
		\par
		\caption{Diegetic Interface \cite{Iacovides2015}.}
		\label{fig:Diegetic}
	\end{minipage}
	\hfill
	\begin{minipage}[!ht]{6cm}
		\woopic{non-diagetic}{.9}
		\par
		\caption{Non-diegetic Interface \cite{Iacovides2015}.}
		\label{fig:Non-Diegetic}
	\end{minipage}
\end{figure}Figure~\ref{fig:Diegetic} shows Battlefield 3 with a diegetic interface and Figure~\ref{fig:Non-Diegetic} shows Battlefield 3 with a non-diegetic interface. Figure~\ref{fig:Diegetic} is diegetic because it does not have 2D images on the screen like Figure~\ref{fig:Non-Diegetic} does. In Figure~\ref{fig:Diegetic}, the bottom right and left hand corners do not have boxes of information and there are not 2D indicators around the world displaying information. as Figure~\ref{fig:Non-Diegetic} is non-diegetic because it does have those 2D displays of information. In one study, subjects experienced in gaming played the same game, Battlefield 3, two times. One time they played it with a diegetic interface, shown in Figure~\ref{fig:Diegetic} and once with a non-diegetic interface, shown in Figure~\ref{fig:Non-Diegetic}. This study found the subjects felt more immersed and engaged in with the diegetic interface. One of the subjects stated "I preferred the challenge of the first game...everything was kind of hidden as well so I had to figure it out for myself. So I seemed to get more involved in it I think" \cite{Iacovides2015} and another said "I did feel that I felt a bit more immersed in the version without the interface simply because there were no flashy things around. So compared to the other one where I was constantly looking at the screen and darting between everything, so I was able to focus on the actual gameplay more". \cite{Iacovides2015} The experimenters then performed another study in which novices and experienced players played the same game in diegetic and non-diegetic forms. The novice players tended to enjoy the non-diegetic interface better because it gave them more information and instruction on how to play. The experienced players did not share this frustration because they were already aware of how to play. Perhaps a non-diegetic interface should have the option to be removed after a user has had adequate experience in playing the game. The removal of non-diegetic interfaces allows for a more immersive and cognitively challenging experience for experienced users \cite{Iacovides2015}. 


Spatial user interfaces are another commonly used interface in virtual reality worlds. It is a good combination of non-diegetic and diegetic interfaces. Spatial interfaces consist of information that is placed inside the world, for example a 3D number floating above the users gun could indicate the amount of ammunition left in the gun. The floating number would be attached to the object so it would move with the gun as the user interacts with it. An example of this can be seen in Figure~\ref{fig:spatialUI}.
\begin{figure}[!ht]
	\begin{center}
		\woopic{spatialUI}{.6}
	\end{center}
	\caption{An example of spacial user interface in VR, photo from unity3d.com/learn/tutorials/topics/virtual-reality/user-interfaces-vr} \label{fig:spatialUI}
\end{figure}
Spacial user interfaces became necessary in the 1990s. This is because as computer tasks became more complex, users began needing a multitude of windows open on the screen. At the time, most user interfaces were non-diegetic and 2D. Ways to combat the cluttered 2D windows on the screen began with the use of task bars and a "window list". These were helpful in organizing the windows but it was still not enough. This is when spatial interfaces began to come into existence. One of the first 3D interfaces was patented in 2006. It consisted of a sphere that is lined with icons linked to different windows. Because of it's 3D shape, not all windows can be viewed at once but it allows for the windows to be located in a small 3D sphere instead of all over the screen. One disadvantage to the sphere is that the user may have problems deciding a logical placement for all of the icons on the sphere \cite{Roca2006}. Spatial interfaces allow for more information and instruction to be given to the user without cluttering the screen and breaking their immersion as much as non-diegetic interfaces.

\section{Conclusion}


