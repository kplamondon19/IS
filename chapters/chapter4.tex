%!TEX root = ../username.tex
\chapter{Archaeology}
Archaeology is the study of humans throughout time. Archaeologist learn about the people of the past by discovering and analyzing objects they left behind. This study focuses on period of the Neanderthals. This area of study is often refereed to as the study of pre-history because most objects found during this time predate modern humans. This project is strongly influenced by an archaeological site called Cova Gran de Santa Linya. Cova Gran is a rock shelter located in Lleida, Spain. Archaeological evidence found at the site has been from 50,000 years ago.    

\section{Who were the Neanderthals}
Neanderthals (the 'th' pronounced as 't') began appearing roughly 200,000 to 100,000 years ago. They continued to exist until about 30,000 years ago when they vanished. Modern humans appeared 40,000 years ago, this means there was approximately 10,000 year overlap between modern humans and Neanderthals. There is much speculation as so why Neanderthals did not survive while modern humans flourished and whether or not Neanderthals and modern humans had any interactions \cite{Smithsonian}.


Neanderthals lived in groups and took care of each other. This is evident from Neanderthal skeletons that are found with bones that were broken and healed within the Neanderthals lifetime. Skeletons have also shown signs of the Neanderthal being toothless for a significant period of time before death. These examples imply that during recovery or before death, these Neanderthals would have needed assistance to live. 


There is also evidence that Neanderthals were capable of speech and the use of symbolism. Ochre and manganese, "crayons", have been found at Neanderthal sites implying their use \cite{Harvati2010}. There is some evidence that Neanderthals may have buried their dead, although it is in no way proven. A Neanderthal skeleton has been found with flowers around it. It is unclear if these flowers were deliberately placed or blown by wind. It is thought that there is a high mortality rate among you and prime age Neanderthals due to quantity and age of found remains.    
 

There are many theories as to why the Neanderthals did not survive. One hypothesis is that Neanderthals assimilated with modern humans. The Neanderthals could have been integrated, assimilated, into modern human groups. They also could have lived separately still sharing technology or trading with each other. It is possible that Neanderthals and modern humans met, worked, or lived together. Another hypothesis is Neanderthals became extinct due to indirect competition with modern humans. Modern humans' advantages may have included larger group sizes, slightly higher birth rates, lower mortality rates, shorter interbirth spacing, greater dietary diversity, more complex social networks, and better clothing and shelter \cite{Harvati2010}. 


\subsection{Appearance}
Today there is a common misconception about the Neanderthal appearance. In 1909 the common belief was that Neanderthals were stupid and crude in comparison to modern human ancestors. 
\begin{figure}
	\centering
	\begin{minipage}{.5\textwidth}
		\centering
		\woopic{1909}{.83}
		\caption{A depiction of Neanderthals from 1909 \cite{larsen_2017}.} \label{fig:Neanderthal-1909}
	\end{minipage}%
	\begin{minipage}{.5\textwidth}
		\centering
		\woopic{neanderthal_Recon_Head}{.26}
		\caption{A reconstruction of a Neanderthal head from present time \cite{Smithsonian}.} \label{fig:Neanderthal-recon}
	\end{minipage}
\end{figure}They were believed to be covered in hair and appear more ape-like than human, shown in Figure~\ref{fig:Neanderthal-1909}. In media today Neanderthals are depicted as brutish cave men however this is an outdated portrayal. Archaeologist no longer believe this description of Neanderthals to be true. Neanderthals are now thought to have appeared more similarly to humans than apes and have a comparable brain size \cite{larsen_2017}.


Neanderthal bodies were designed to withstand the cold. They had large nasal openings which allowed for more air to be heated before entering the lungs. The brain size of Neanderthals is larger than most modern humans. A reconstruction of a Neanderthal head is shown in Figure~\ref{fig:Neanderthal-recon}. They had relatively short limbs and a short stature. 
Neanderthal males were an average height of 5 ft 5 in, females were an average height of 5 ft 1 in. The average weight for male Neanderthals was 143 lbs, the average weight for a female was 119 lbs \cite{Smithsonian}. The short stature and short limbs are classic adaptations to the cold. Less skin surface area means less energy to heat. 


\subsection{Tools}
The Neanderthals used many types of tools to help them achieve everyday tasks. To protect themselves from the elements Neanderthals made clothing and lived in shelters. They would create tools for sewing such as needles and awls. They used fire for its warmth, protection from predators, cooking, light, tool tempering, vegetation clearing, hunting, preserving food, and pest control. The size, shape, and placement of an excavated hearth, floor of a fire place, can help explain what the fire's use was. Hearths that are found next to the rock walls may have been used to warm the rock and help heat a sleeping area. Hearths that have holes on either side of them, presumably where sticks were placed vertically, may have been used for cooking. Some hearths have been found with large rocks on them, this may have been done by the Neanderthals to stop the use of the hearth. The inspection hearths, charcoal, can also indicate what kind of wood was burned, what was being cooked, and the temperature of the fire. Hearths that held high temperature fires indicated that the users of the hearth had some kind of understanding of how fires can be created and maintained. This information can help archaeologist speculate as to what they know and techniques the Neanderthals may have used.  


Neanderthals also made and used stone tools. In archaeology a lithic is a stone that has been manipulated by a  human ancestor or a Neanderthal. Tools were used in many activities such as hunting, cleaning hide and cutting wood and food. This is evident by lithic material being found in hide, animal horns, and in wood. Tools were not only created but also specialized. Different activities required different tools. 


Neanderthals hunted in groups and often used planned techniques to hunt successfully. There have been some well-preserved wooden spears dating 400,000 years ago \cite{Harvati2010}. It is uncommon to find wooden artifacts, such as these spears, because wood needs specific circumstances to become preserved. It is likely other wooden tools were made and used by the Neanderthals. It is hypothesized that Neanderthals did not throw these spears but instead would use them with a pushing motion. This is thought to be true because of their upper body bone structure. They would not easily be able to put their arms above their head and would have a stronger force behind the pushing rather than throwing motion. They also lead animals off cliffs or make traps to catch prey. Hunting however was extremely risky and required a lot of energy and calories. Some Neanderthal skeletons have been found having similar injures to that of bull riders. It was more common for Neanderthals to scavenge than to hunt. If they found a carcass it was most likely mostly cleaned by birds with the bones being left behind. To extract nourishment from the bones, Neanderthals would use stones to beak open the bone and eat the marrow from inside.


\subsubsection{Tool Classification}
To make stone tools Neanderthals used a method we have named knapping. Knapping is the action of hitting one rock, called the hammer stone, against another rock. Flit was a common material used to make tools because when hit, sharp flakes break off. Quartzite is a bit harder to knap but easier to find. Limestone has also been used but it very easily broken. While knapping there debris falls off of the flint as well as the intended tool. 
\begin{figure}[!ht]
	\begin{center}
		\woopic{neanderthal-tools}{.26}
	\end{center}
	\caption{Neanderthal toolkit : https://www.amnh.org/exhibitions/permanent-exhibitions/anne-and-bernard-spitzer-hall-of-human-origins/neanderthal-tools. } \label{fig:Neanderthal-tool}
\end{figure}The hammer stone is usually round and made from a material harder than the stone being knapped and is usually discovered intact. Neanderthals could strike the flint in deliberate ways to create different tools. After a flake could then be further developed with retouching. Retouching is when the flake is knapped to create something new. A round flake, as seen in Figure~\ref{fig:Neanderthal-tool}, is called an end scraper and can be used to clean hide. Blades are taller that they are wide, these can be used in hunting. A notched scrapper, shown in Figure ~\ref{fig:Neanderthal-tool}, is used to take the bark off of pieces of wood. It would be used by placing the wood in the notch and scraping. Some stones were made for chopping wood and digging. After a piece of flint is too small to strike again it was usually discarded by the Neanderthals, this discarded piece is called the core. 

\subsection{Diet}
Neanderthal sites have been found with many animal remains indicating meat being apart of their diet. Animals that have been commonly found at Neanderthal sites include bison, wild cattle, horse, reindeer, red and fallow deer, ibex, wild boar, and gazelle. It is even thought that larger animals such as wholly rhinoceros or wholly mammoth were consumed by Neanderthals \cite{Harvati2010}. At some sites there has been remains of shellfish, birds, and marine mammals \cite{Harvati2010}. Neanderthals did eat plants but most of their diet was made up of meat. 


\section{Cova Gran}
Cova Gran is an interesting site because of the period at which artifacts have been dated. Cova Gran has artifacts ranging from before Neanderthals vanished, to the overlapping of Neanderthals and modern humans (40,000 - 30,000 years ago), to after Neanderthals vanished. Cova Gran is a rock shelter, not a cave, meaning it is a rock overhang that can give protection from the elements. It is thought to have been used as a temporary living space for nomads or shepherds throughout the years. Today Cova Gran's opening is much larger than it would have been during the time of the Neanderthals due to the rock ceiling falling gradually over time.    
\begin{figure}[!ht]
	\begin{center}
		\woopic{CovaGrab}{.5}
	\end{center}
	\caption{Photo of the Cova Gran de Santa Liny field site \cite{CovaGran}.} \label{fig:CovaGrab}
\end{figure}  


Every summer a team of archaeologists and students work on excavation at Cova Gran. For an average archaeologist, who's job is to excavate the site, most days consist of excavation, cleaning, analyzing, and organizing artifacts. 

\section{Game Topics}
This projects covers some topics of a Neanderthal and archaeologists life. Players learn about how Neanderthal made ad used tools, what they ate and hypotheses as to why they vanished. Players also learn about what it is like excavating at an archaeological dig site.  
 



%how Neanderthals build a fire I was thinking of how I could implement fire building into the game. 
%First the player would need to gather all the materials needed for the kind of fire making they 
%wanted to use. If they were using the percussion technique they would need to gather the 
%correct type of stones, tinder such as a small bundle of dried grass, and some larger sticks to get 
%burning. Once those materials were collected they could have the action ability to hit the rocks 
%together over the tinder and make sparks. In the code for this bit I could have likelihood that the 
%tinder would catch, a 50 percent chance of the tinder lighting would work well. Then once the fire 
%was lit that activity would be complete. 


\section{Conclusion}


%FROM ESSAY

%The tools used in excavation include a rubber bucket, a screwdriver, a box of thumb pins, multiple small bags, a paint brush, a small shovel, small tubes, a wooden stick, two sitting mats, and labels. 
%Once gathering the tools, I would go to the site and remove the tarp covering the pit and 
%sit in my designated area. The directors decide what areas are to be worked on that day and then
%the work is divided between the people working that day. It is important to keep the entire pit at 
%roughly the same level and not have areas that are significantly higher or lower than the other. 
%Once I know what area I am responsible for that day I set my mats down and lay out my 
%supplies. Before I begin digging I look at the area and decide where I need to start first. In the 
%specific area I was working there was a slope in the level, the bottom of the slope was in the 
%south east corner, it is also important to be in contact and work hand in hand with the people 
%around you so that the slope is consistent between everyone in the pit.  
%Once deciding where I should begin to dig, finding where the slope should be and 
%talking to the people around me I can begin to make the first marks in the soil. I begin by 
%cleaning the area but brushing any debris that has gathered into my trowel and emptying the 
%contents of the trowel into the rubber bucket. Once the area is clean I take the screwdriver and 
%press into the soil at a 45-degree angle which in turn breaks up the soil and loosens the dirt. I 
%work like a typewriter going from left to right working in rows downward until I reach the 
%bottom of my work area. Once a thin layer of soil is loosened I use the large paint brush to 
%gently push the loose dirt into the small trowel. The dirt from the trowel is then placed into the 
%rubber bucket, this activity is referred to as cleaning. If during the cleaning process any piece of 
%artifact is revealed then a push-pin is placed next to the artifact so it is easy to see and not loose 
%track of. The artifact is then dug around to make sure is not damaged and is not picked up until 
%it looks like it is no longer in the ground, this is done to prevent holes being created in the layer. 
%When digging closer to the artifact I used a wooden stick to release it from the dirt as to not 
%damage the artifact.  When the artifact is clear from the ground it is given a label. Each level has different 
%labels, for example when digging in level V-12 each artifact's label will say V-12 on it with the corresponding code for V-12 as well as an individual number and code so it can be identified later on. I then take the coordinate of the object, this is done by using the prism and the total station transit. I tell the person controlling the total station what level I am working on, I then place the tip of the prim next to the artifact and make sure it is vertical using a bubble level that is built into the prism. When the bubble is steady and prism is vertical I yell the number of the 
%artifact and wait for the person controlling the total station to tell me they successfully got the 
%point. It was important for me to clearly say the number of the piece because it is easy to mix 
%things up so many double checks are in place, if I say a number out of turn then the person 
%controlling the total station can correct me and I can find the correct piece as to not mix all the 
%artifacts locations up. After the artifacts location is placed then I put the artifact and its label 
%into a small plastic bag and tie it loosely just s the artifact will not fall out. At Cova Gran 
%artifacts are commonly either lithics, small animal bones, or charcoal. If a material is too small 
%or insignificant to get a label and point to itself it is placed in a no coordinate bag along with 
%others of its kind. When the rubber bucket has a few cleanings worth of soil it is taken to the sift. The 
%content of the rubber bucket is place onto a screen and shaken to be rid of the excess dirt, then 
%the screen is placed into a large bucket of water and shaken to clean the remaining materials. 
%Once the materials are clean it is easier to see what everything is and it is important to look to 
%see if you missed any artifacts. It is common to find no coordinates in the sifting process due to 
%their smaller size. If there is something important found in the sift it is brought back to the site 
%and randomly placed in the area you were working to be given a label and a coordinate.   
%After excavation the lab work begins. Lab work entails cleaning, labeling and categorizing. In the field the materials for each level are placed in a larger "mother" bag this bag is then transported back to the lab to begin the next steps of the process. The first thing that is done to the materials in the lab is cleaning, to clean materials a shallow tub of water is placed in the middle of the table. While cleaning it is important to keep the materials from different levels separate as to cause less trouble down the road. 


%The first step of cleaning is opening all the bags, then all the materials and their corresponding labels are placed on the table in row. Then with one hand the material is picked up and with the other hand it is placed in the water and 
%rubbed with the fingers to remove any dirt or debris. To place the material back on the table the 
%other hand is used to decrease the amount of water on the material and table which in turn 
%decreases drying time. This is done to all the materials in the rows for that level, once they are 
%done then the no coordinates can be cleaned. No coordinates are cleaned by being placed in a 
%small sift and placed into the water, if any larger materials are inside then they can be cleaned 
%by hand. Charcoals are kept in their tubes and not cleaned, they are placed in a separate area 
%with their caps off allowing them to "breath". There are some pieces that should not be cleaned 
%because they are either to fragile and would fall apart in the cleaning process or they would lose 
%some information that is on the exterior of the material. For example, hammer stones are not 
%washed because they can be analyzed to see what is on them which can lead to a better 
%understanding for what it was used for.  


%Once all the materials are dry it is time for the labeling processes. To label the materials 
%you need a pair or tweezers, clear nail polish, and a wooden stick. On the labels that have been 
%put with the materials there are three sizes of QR codes. These QR codes are linked to a number 
%in the database which stores all the data about the specific material. To access this information 
%all that needs to be done is scan the QR code and the information for the material is accessed 
%and displayed this information can be added to and edited if need be. The first numbers linked 
%to the QR codes show that the material was found at Cova Gran, the next numbers correspond 
%to the level the material was found, and the last numbers correspond to the specific number 
%given to the material. The smaller codes only have the last bit of information, so it is better to 
%use the larger codes when possible. To place a QR code onto an artifact the first step is to 
%decide where the code should go, if it is a lithic it should be placed on the ventral side as far 
%from the edge as possible. The goal of placing QR codes is to link the material to the database 
%and not cover up important information about the piece such as burnt areas or a place where the 
%piece was retouched. It is also important to place the code on a clean, smooth, and flat surface 
%to prevent the code from falling off. In some cases, a wooden stick can be used to scrape away 
%any concretion that is in the way of optimal placement of the QR code. When the proper 
%placement is found a dab of the clear nail polish is placed in that location. It is important to 
%make sure that the nail polish is not too thick because if that is the case it is probable that it will 
%dry bubbly and the code would not scan and the material would need to have a new code placed 
%on it. After the first coat of nail polish is on then with tweezers the QR code is chosen and 
%placed in that area. It is important to make sure the entire code was printed on the sticker, so it 
%will be able to be scanned later on. Then a second layer of nail polish is placed on top of the 
%sticker with the code on it to keep it in place. Once all the materials in the level are marked with 
%their code they are placed in a shallow box and each one is scanned to make sure the QR code 
%works and shows the correct number. After this is done they are ready to be classified.  
%To begin the classification, process a single level is chosen, this level will be completed 
%before moving to a different level to avoid mixing the levels. Smaller boxes are gathered and 
%are designated to hold one type of material. 




