%%%%%%%%%%%%%%%%%%%%%%%%%%%%%%%%%%%%%%%%%%%%%%%%%%%%%%%
%
%                                                       Example IS Template
%
% \documentclass{woosterthesis} must be at the beginning of every IS. Options are the same as
% for the report class with some additional options, abstractonly, blacklinks, code, kaukecopyright, palatino, picins,
% maple, index, verbatim, dropcaps, euler, gauss, alltt,  woolshort, colophon, woosterchicago, and
% achemso. The kaukecopyright option will put the arch symbol with the word mark on the
% copyright page. The woosterthesis class is based on the report class. One thing to note is that
% the ``%'' symbol comments out all characters that follow it on the line.
%
%%%%%%%%%%%%%%%%%%%%%%%%%%%%%%%%%%%%%%%%%%%%%%%%%%%%%%%

%%%%%%%%%%%%%%%%%%%%%%%%%%%%%%%%%%%%%%%%%%%%%%%%%%%%%%%
% use this declaration for a draft  version of your IS
\documentclass[10pt,palatino,code,picins,kaukecopyright,openright,woolshort,dropcaps,verbatim,index,euler]{woosterthesis}
%\documentclass[10pt,code,picins,kaukecopyright,openright,woolshort,dropcaps,verbatim,euler,index,colophon,blacklinks,twoside]{woosterthesis}
% note that you can specify the woosterchicago option to use Chicago citation style and achemso to use the American Chemical Society citation format
%
%%%%%%%%%%%%%%%%%%%%%%%%%%%%%%%%%%%%%%%%%%%%%%%%%%%%%%%
%
% use this declaration for the print version of your IS
%\documentclass[12pt,code,palatino,picins,blacklinks,kaukecopyright,openright,twoside]{woosterthesis} % probably what most students would use
%
%%%%%%%%%%%%%%%%%%%%%%%%%%%%%%%%%%%%%%%%%%%%%%%%%%%%%%%
%
% use this declaration for the PDF version of your IS
%\documentclass[12pt,code,palatino,picins,kaukecopyright,openright,twoside]{woosterthesis}
%
%%%%%%%%%%%%%%%%%%%%%%%%%%%%%%%%%%%%%%%%%%%%%%%%%%%%%%%

%%%%%%%%%%%%%%%%%%%%%%%%%%%%%%%%%%%%%%%%%%%%%%%%%%%%%%%
%
%                                                       Load Packages
%
%   To load packages in addition to the ones that are loaded by default, please place your
%   usepackage commands in the packages.tex file in the styles folder.
%
%%%%%%%%%%%%%%%%%%%%%%%%%%%%%%%%%%%%%%%%%%%%%%%%%%%%%%%

%%%%%%%%%%%%%%%%%%%%%%%%%%%%%%%%%%%%%%%%%%%%%%%%%%%%%%%%%%%%%%%%%%%%%%%%%%%%%%%%%%%%%%%%%%%%%%
%
%                                                       Packages
%
% Do not add any other packages without consulting with Dr. Breitenbucher as they may break the functionality of the class.
%
%%%%%%%%%%%%%%%%%%%%%%%%%%%%%%%%%%%%%%%%%%%%%%%%%%%%%%%%%%%%%%%%%%%%%%%%%%%%%%%%%%%%%%%%%%%%%%

\ifxetex%
	\defaultfontfeatures{Mapping=tex-text}%
		\setmainfont[Numbers=OldStyle,BoldFont={* Semibold}]{Adobe Garamond Pro}% select the body font other choices would be Baskerville, Optima Regular, Didot, Georgia, Cochin
                      \setmathrm{Adobe Garamond Pro}
                      \setmathfont[Digits,Latin]{Adobe Garamond Pro}
		\setsansfont[Scale=.87,Fractions=On,Numbers=Lining]{Myriad Pro}% select the sans serif font other choices would be Skia, Arial, Helvetica, Helvetica Neue
%		\setmonofont[Scale=.88,Fractions=On]{Prestige Elite Std Bold}% set the mono font other choices would be Courier, Monaco, American Typewriter
	           \setmonofont[Scale=.9]{Courier Std}%
%	    \setromanfont[Fractions=On,Numbers=OldStyle, BoldFont={Warnock Pro Semibold}]{Warnock Pro}%
%	    \setsansfont[Scale=.95,Fractions=On,Numbers=Lining]{Myriad Pro}%
%	    \setmonofont[Scale=.91,Fractions=On]{Courier Std Medium}%
%	    \setmonofont[Scale=.88,Fractions=On]{American Typewriter}%
%		\setmonofont[Scale=.94,Fractions=On]{Prestige Elite Std Bold}
%    		\setromanfont[Fractions=On,Numbers=OldStyle]{Minion Pro}
 %    	\setsansfont[Scale=.9,Fractions=On,Numbers=Lining]{Myriad Pro}
%     	\setmonofont[Scale=.93,Fractions=On]{Courier Std Medium}
%     	\setromanfont[Fractions=On,Numbers=OldStyle]{Minion Pro}
%     	\setsansfont[Scale=.85,Fractions=On,Numbers=Lining]{News Gothic Std}
%    		\setmonofont[Scale=.93,Fractions=On]{Prestige Elite Std}
%		\setromanfont[Fractions=On,Numbers=OldStyle]{Minion Pro}
%		\setsansfont[Scale=.9,Fractions=On,Numbers=Lining]{Bell Gothic Std Bold}
%		\setmonofont[Scale=.95,Fractions=On]{Prestige Elite Std Bold}
\fi

%%%%%%%%%%%%%%%%%%%%%%%%%%%%%%%%%%%%%%%%%%%%%%%%%%%%%%%
%
%                                                       Load Personal commands
%                                                                    
%  There will be certain commands that you use frequently in the thesis. You can give these
%  commands new names which are easier for you to remember. You can also combine several
%  commands into a new command of your own. See The LaTeX Companion or Guide to LaTeX
%  for examples on defining your own commands. These are commands that I defined to cut
%  down on typing. You can enter your commands in the personal.tex file in the styles folder.
%
%%%%%%%%%%%%%%%%%%%%%%%%%%%%%%%%%%%%%%%%%%%%%%%%%%%%%%%

%%%%%%%%%%%%%%%%%%%%%%%%%%%%%%%%%%%%%%%%%%%%%%%%%%%%%%%%%%%%%%%%%%%%%%%%%%%%%%%%%%%%%%%%%%%%%%
%
%                                                       Personal Commands
%                                                                    
% There will be certain commands that you use frequently in the thesis. You can give these
% commands new names which are easier for you to remember. You can also combine several
% commands into a new command of your own. See The LaTeX Companion or Guide to LaTeX for
% examples on defining your own commands. These are commands that I defined to cut down on typing.
%
%%%%%%%%%%%%%%%%%%%%%%%%%%%%%%%%%%%%%%%%%%%%%%%%%%%%%%%%%%%%%%%%%%%%%%%%%%%%%%%%%%%%%%%%%%%%%%

\newcommand{\fl}{\ell}
\newcommand{\lt}{\LaTeX\ }
\newcommand{\msw}{Word\texttrademark\ }
\newcommand{\xt}{\ifthenelse{\boolean{xetex}}{\XeTeX\ }{XeTeX} }
%\newcommand{\Cl}{\ensuremath{\textup{C}_\fl}}
%\newcommand{\bCl}{C$_{\ell}$}
%\newcommand{\Al}{\ensuremath{\textup{A}_\fl}}
%\newcommand{\msum}{{(m_1+\cdots+m_\ell)}}
%\newcommand{\Nsum}{{(N_1+\cdots+N_\ell)}}
%\newcommand{\ysum}{{(y_1+\cdots+y_\ell)}}
%\newcommand{\Nsub}{{N_1+\cdots+N_\ell}}
%\newcommand{\ysub}{{y_1+\cdots+y_\ell}}
%\newcommand{\xsub}{{x_1+\cdots+x_\ell}}
%\newcommand{\ysqsum}{{y_1^2+\cdots +y_{\fl}^2}}
%\newcommand{\msqsum}{{m_1^2+\cdots +m_{\fl}^2}}
%\newcommand{\ratio}{\left(\frac{\beta}{\alpha}\right)}
%\newcommand{\LT}{\ensuremath{\LaTeX{}}}

%%%%%%%%%%%%%%%%%%%%%%%%%%%%%%%%%%%%%%%%%%%%%%%%%%%%%%%%%%%%%%%%%%%%%%%%%%%%%%%%%%%%%%%%%%%%%%
% These commands have one argument and are entered as \commandname{argument}.
%%%%%%%%%%%%%%%%%%%%%%%%%%%%%%%%%%%%%%%%%%%%%%%%%%%%%%%%%%%%%%%%%%%%%%%%%%%%%%%%%%%%%%%%%%%%%%

%\newcommand{\bd}[1]{\textbf{#1}}
\newcommand{\mbd}[1]{{\mathbf{#1}}}
%\newcommand{\abs}[1]{\vert{#1}\vert}
\newcommand{\bvec}[1]{{\mbd{#1}}}
%\newcommand{\lvec}[1]{\abs{\bvec{#1}}}
%\newcommand{\nesmallprod}[1]{\prod_{\substack{#1=1\\
%#1\neq p}}^{\fl}}
%\newcommand{\esec}[1]{e_{2}({#1}_1,\ldots ,{#1}_\fl)}
%\newcommand{\smallprod}[1]{\prod_{#1=1}^{\fl}}
%\newcommand{\incsum}[1]{{#1}_2+2{#1}_3+\cdots +(\fl -1){#1}_\fl}
%\newcommand{\binomsum}[1]{\binom{{#1}_1}{2}+\cdots +\binom{{#1}_\fl}{2}}
%\newcommand{\imultsum}[1]{\multsum{{#1}_k\ge 0}{k=1,\ldots ,\fl}}
%\newcommand{\diagsum}[1]{\sum _{\substack{{#1}_k\ge 0\\
%k=1, \ldots ,\fl\\
%\lvec{#1}=m}}}
%\newcommand{\Mb}[1][\fl]{\ensuremath{\textup{\bd{M}}_b^{(#1)}}}
%\newcommand{\HLV}[1]{\ensuremath{\textup{\bd{H}}_{#1}}}
%\newcommand{\Rq}[1][p]{\ensuremath{\textup{R}_q^{(#1)}}}
\newcommand{\degree}[1]{\ensuremath{#1^{\circ}}}
\newcommand{\ip}[1]{\texttt{#1}\index{packages!#1}}
\newcommand{\ic}[1]{\texttt{$\backslash$#1}\index{commands!#1}}
\newcommand{\ie}[1]{#1\index{#1}}

%%%%%%%%%%%%%%%%%%%%%%%%%%%%%%%%%%%%%%%%%%%%%%%%%%%%%%%%%%%%%%%%%%%%%%%%%%%%%%%%%%%%%%%%%%%%%%
% These commands have 2 or more arguments some with default values for the first argument. You
% can learn a lot about constructing complicated equations by studying the commands in this %section.
%%%%%%%%%%%%%%%%%%%%%%%%%%%%%%%%%%%%%%%%%%%%%%%%%%%%%%%%%%%%%%%%%%%%%%%%%%%%%%%%%%%%%%%%%%%%%%

%\newcommand{\qbinom}[2]{\ensuremath{\left[{#1}\atop{#2}\right]_q}}
%\newcommand{\sqprod}[2]{\prod_{#1,#2=1}^{\fl}}
%\newcommand{\triprod}[2]{\prod_{1\le #1<#2\le \fl}}
%\newcommand{\nesqprod}[2]{\prod_{\substack{#1,#2=1\\
%#1,#2\neq p}}^{\fl}}
%\newcommand{\netriprod}[2]{\prod_{\substack{1\le #1<#2\le \fl\\
%#1,#2\neq p}}}
\newcommand{\qrfac}[3][\ ]{\left({#2}\right)_{#3}^{#1}}
%\newcommand{\multsum}[2]{\sum_{\substack{{#1}\\
%\\
%{#2}}}}
%\newcommand{\fmultsum}[2][N]{\multsum{0\le {{#2}_k}\le {{#1}_k}}{k=1,\ldots ,\fl}}
%\newcommand{\pq}[2]{\ _{#1}\varphi_{#2}}
%\newcommand{\mess}[2][y_k]{\frac{\qrfac{\alpha x_k}{#2}\qrfac{qx_k\beta^{-1}}{#2}}{\qrfac{\beta x_k}{#1}
%\qrfac{qx_k\alpha^{-1}}{#1}}}
%\newcommand{\MG}[7][\fl]{\ensuremath{\left[\textup{MG}\right]_{#2}^{(#1)}{#3}q;{#4};{#5}^{#6}{#7}}}

%%%%%%%%%%%%%%%%%%%%%%%%%%%%%%%%%%%%%%%%%%%%%%%%%%%%%%%%%%%%%%%%%%%%%%%%%%%%%%%%%%%%%%%%%%%%%%
% These commands define new environments
%%%%%%%%%%%%%%%%%%%%%%%%%%%%%%%%%%%%%%%%%%%%%%%%%%%%%%%%%%%%%%%%%%%%%%%%%%%%%%%%%%%%%%%%%%%%%%

\newcounter{unnumft}
\setcounter{unnumft}{0}
\newenvironment{unnumft}[2]{\renewcommand{\thefootnote}{}\footnote{#1}\footnote{#2}} {\addtocounter{footnote}{-2}}
\newenvironment{wooexample}{\small
\begin{singlespace}
\begin{example}}{\end{example}
\end{singlespace}}

\graphicspath{{./figures/}}% for setting where to look for figures
\citestyle{wooster}% change the style of citations. Math and CS people should leave this alone.






%%%%%%%%%%%%%%%%%%%%%%%%%%%%%%%%%%%%%%%%%%%%%%%%%%%%%%%
%
%                                                       Load Theorem formatting information
%
%  If you need to define an new theorem style or want to see what theorem like environments 
%  are available please look at the theorems.tex file in the styles folder.
%
%%%%%%%%%%%%%%%%%%%%%%%%%%%%%%%%%%%%%%%%%%%%%%%%%%%%%%%

%%%%%%%%%%%%%%%%%%%%%%%%%%%%%%%%%%%%%%%%%%%%%%%%%%%%%%%%%%%%%%%%%%%%%%%%%%%%%%%%%%%%%%%%%%%%%%
%
% This is where one would tell \LaTeX{} how to format Theorems, Definitions, etc. and also
% indicate the environment names. You need the amsthm package (loaded in the woosterthesis %class) in order for these commands to work.
%
%%%%%%%%%%%%%%%%%%%%%%%%%%%%%%%%%%%%%%%%%%%%%%%%%%%%%%%%%%%%%%%%%%%%%%%%%%%%%%%%%%%%%%%%%%%%%%

% an example of defining your own theoremstyle
%\newtheoremstyle{break}% name
%  {\topsep}%      Space above
%  {\topsep}%      Space below
%  {\itshape}%         Body font
%  {}%         Indent amount (empty = no indent, \parindent = para indent)
%  {\bfseries}% Thm head font
%  {.}%        Punctuation after thm head
%  {\newline}%     Space after thm head: " " = normal interword space;
%        %       \newline = linebreak
%  {}%         Thm head spec (can be left empty, meaning `normal')
\newtheoremstyle{scthm}{\topsep}{\topsep}{\itshape}{}{\bfseries\scshape}{}{ }{}% small cap font for the heading
\newtheoremstyle{itdefn}{\topsep}{\topsep}{\itshape}{}{\bfseries}{.}{ }{}% italic definitions
\newtheoremstyle{scdefn}{\topsep}{\topsep}{\itshape}{}{}{}{ }{\thmname{\textbf{#1}}\thmnumber{ \textbf{#2}}\thmnote{ \scshape #3:}}% small cap headings and italic text.

\theoremstyle{break}% this theoremstyle will put the text of the theorem on a new line.
\newtheorem{thm}{Theorem}[chapter]%number theorems within chapters 
%\newtheorem{cor}[thm]{Corollary}%by using [thm] we are numbering these environments with the theorems.
\newtheorem{cor}{Corollary}[chapter]%number corollaries within chapters .
%\newtheorem{lem}[thm]{Lemma}
\newtheorem{lem}{Lemma}[chapter]
%\newtheorem{prop}[thm]{Proposition}
\newtheorem{prop}{Proposition}[chapter]

\theoremstyle{scdefn}
%\newtheorem{defn}[thm]{Definition}
\newtheorem{defn}{Definition}[chapter]
\theoremstyle{remark}
%\newtheorem{rem}[thm]{Remark}
\newtheorem{rem}{Remark}[chapter]
\renewcommand{\therem}{}
%\newtheorem{ex}[thm]{Example}
\newtheorem{ex}{Example}[chapter]

\theoremstyle{plain}
%\newtheorem{note}[thm]{Notation}
\newtheorem{note}{Notation}[chapter]
\renewcommand{\thenote}{}
%\newtheorem{nts}[thm]{Note to self}%use to remind yourself of things yet to do
\newtheorem{nts}{Note to self}[chapter]
\renewcommand{\thents}{}
%\newtheorem{terminology}[thm]{Terminology}
\newtheorem{terminology}{Terminology}[chapter]
\renewcommand{\theterminology}{}

\theoremstyle{itdefn}
\newtheorem{bdefn}{Definition}[chapter]
\newsavebox{\fmbox} 
\newenvironment{boxeddefn}[2] 
{\begin{lrbox}{\fmbox}\begin{minipage}{0.9 \linewidth }\begin{singlespace}\begin{bdefn}[{#1}]\label{#2}\vspace{0.2cm}} 
{\end{bdefn}\end{singlespace}\end{minipage}\end{lrbox}\fbox{\usebox{\fmbox}}}

\setcounter{secnumdepth}{5}% controls the numbering of sections
\setcounter{tocdepth}{6}% controls the number of levels in the Contents

%%%%%%%%%%%%%%%%%%%%%%%%%%%%%%%%%%%%%%%%%%%%%%%%%%%%%%%
%
%  This is where one enters the information about the thesis.
%
%%%%%%%%%%%%%%%%%%%%%%%%%%%%%%%%%%%%%%%%%%%%%%%%%%%%%%%


\title{Game-Based Learning in Virtual Reality}
\thesistype{Independent Study Thesis} % you should make this Independent Study Thesis
\author{Kz Plamondon}
%\presentdegrees{Ph.D.} % you should comment this line
\degreetoobtain{Bachelor of Arts}
\presentschool{The College of Wooster}
\academicprogram{Department of Mathematics and Computer Science}
\gradyear{2019}
\advisor{Denise Byrnes (Computer Science)}
%\secondadvisor{Second Advisor}
%\reader{Reader}
\copyrighted   
%\copyrightdate{}                  
\makeindex % comment this line if you do not have an index

%%%%%%%%%%%%%%%%%%%%%%%%%%%%%%%%%%%%%%%%%%%%%%%%%%%%%%%
%
%  This is where the commands for the document begin. All \LaTeX{} documents must have a
%  \begin{document} text .... \end{document} structure.
%
%%%%%%%%%%%%%%%%%%%%%%%%%%%%%%%%%%%%%%%%%%%%%%%%%%%%%%%

\begin{document}

%%%%%%%%%%%%%%%%%%%%%%%%%%%%%%%%%%%%%%%%%%%%%%%%%%%%%%%
%
%  The front matter includes acknowledgments, dedications, vitas, list of tables, list of figures,
%  copyright, abstract, title page, and contents.
%
%%%%%%%%%%%%%%%%%%%%%%%%%%%%%%%%%%%%%%%%%%%%%%%%%%%%%%%

\frontmatter
\maketitle
\ClearShipoutPicture
\clearpage\thispagestyle{empty}\null\clearpage
\disscopyright 

%%%%%%%%%%%%%%%%%%%%%%%%%%%%%%%%%%%%%%%%%%%%%%%%%%%%%%%
%                                                                                       
%                                                       Abstract						
%                                                                                       
%%%%%%%%%%%%%%%%%%%%%%%%%%%%%%%%%%%%%%%%%%%%%%%%%%%%%%%

\begin{abstract}
This thesis uses the Oculus Rift and Unity to create a game that can be used to teach students and the public about archeology through game-based-learning.

%The focus of this Computer Science Independent Study is to develop a virtual reality (VR) game that is used to teach students and the public about archeology through game-based-learning. The project will focus on creating an immerse experience with controls that come naturally to the user. The VR hardware that will be used for is project is the Oculus Rift \cite{oculus}. The user's controller in the game will either be developed for the Oculus Touch or the Leap Motion \cite{leap}. The Oculus Touch is a controller with buttons and triggers that can be pressed to initiate some response. The touch allows the user to interact with the environment and receive haptic feedback by way of vibrations. The Leap Motion is attached to the headset of the Oculus and tracks the user's hands allowing hand gestures to be used as controllers. The game's software will be developed in Unity \cite{unity} and if extra 3D models are needed, Blender \cite{blender} or Autodesk 3Ds Max \cite{autodesk} will be used. This project will explore the use of games as a teaching tool, creating immerse VR environments and the creation of VR games and their development.


%Problems that may arise include the controls not syncing with the created project, this would make using the controllers hard if not impossible. The leap motion controller is limited in the range it can be used, the user has to be looking at their hands in order to use their hands as controllers. This might make playing the game difficult and ultimately result in using different controllers such as the Oculus Touch. It may be overambitious to make every detail accurate, such as all of the plants in the terrain.   

%By December 7, 2018 70 percent of the paper and 70 percent of the software will be done.
\end{abstract}

%%%%%%%%%%%%%%%%%%%%%%%%%%%%%%%%%%%%%%%%%%%%%%%%%%%%%%%
%                                                                                       
%                                                       Dedications					
%                                                                                       
%%%%%%%%%%%%%%%%%%%%%%%%%%%%%%%%%%%%%%%%%%%%%%%%%%%%%%%

\dedication{This work is dedicated to the future generations of Wooster students.}


%%%%%%%%%%%%%%%%%%%%%%%%%%%%%%%%%%%%%%%%%%%%%%%%%%%%%%%
%                                                                                       
%                                                       Acknowledgments					
%                                                                                       
%%%%%%%%%%%%%%%%%%%%%%%%%%%%%%%%%%%%%%%%%%%%%%%%%%%%%%%

\begin{acknowl}  
I would like to acknowledge Prof. Lowell Boone in the Physics Department for his suggestions and code.
\end{acknowl}

%%%%%%%%%%%%%%%%%%%%%%%%%%%%%%%%%%%%%%%%%%%%%%%%%%%%%%%
%                                                                                       
%                                                       Vita					
%                                                                                       
%%%%%%%%%%%%%%%%%%%%%%%%%%%%%%%%%%%%%%%%%%%%%%%%%%%%%%%

\begin{vita} 
% You talk about yourself and how you got to where you are now. There is a structured form for the Vita that can be used if you want, but I don't encourage it.

%%%%%%%%%%%%%%%%%%%%%%%%%%%%%%%%%%%%%%%%%%%%%%%%%%%%%%%
%
%  The list below is for a thesis that requires a more structured Vita such as a masters or Ph.D.
%
%%%%%%%%%%%%%%%%%%%%%%%%%%%%%%%%%%%%%%%%%%%%%%%%%%%%%%%

%\begin{datelist}
%\item[April 6, 1970]Born-Wooster, Ohio
%\item[August 11, 1990]Chosen to present an undergraduate paper at the 75th meeting of the MAA, Columbus, Ohio
%\item[August 1990--August 1991]President Wooster Student Chapter of the MAA, The College of Wooster, Wooster, Ohio
%\item[August 1991--May 1992]Secretary Wooster Student Chapter of the MAA, The College of Wooster, Wooster, Ohio
%\item[1992]\emph{Phi Beta Kappa} (on junior standing), The College of Wooster, Wooster, Ohio
%\item[1992]Elizabeth Sidwell Wagner Prize in Mathematics, The College of Wooster
%\item[1992]William H. Wilson Prize in Mathematics, The College of Wooster
%\item[May 11, 1992]B.A., Mathematics, The College of Wooster
%\item[1997]Finalist for Graduate Teaching Award, The Ohio State University, Columbus, Ohio
%\item[June 21-25, 1998]Participant in the AMS-IMS-SIAM Summer Research Conferences: q-Series, Combinatorics, and Computer Algebra, Mt. Holyoke, Massachusetts
%\item[October 1998--October 1999]Graduate student representative to The Ohio State University Department of Mathematics Graduate Studies Committee, Columbus, Ohio
%\item[January 1999]q-series seminar address, The Ohio State University, Columbus, Ohio
%\item[2000]Finalist for Departmental Teaching Award, The Ohio State University, Columbus, Ohio
%\item[2000]Nominated for Graduate Teaching Award, The Ohio State University, Columbus, Ohio
%\item[April 2000]Invited colloquium talk at The College of Wooster, Wooster, Ohio
%\item[1992-- present]Graduate Teaching and Research Associate, The Ohio State University
%\end{datelist}
%
%%%This is for any publications you might have.%%%%%

\begin{publist}  
\pubitem{\quad}
\pubitem{\quad}
\end{publist} 

\begin{fieldsstudy} 
    \majorfield{Computer Science}
	\minorfield{Anthropology}
    %\specialization{Area of IS research} %SHOULD I HAVE THIS
    %\begin{studieslist}
   %\studyitem{Abstract Algebra}{Hampton}
   %\end{studieslist}
  \end{fieldsstudy}
\end{vita}

%%%%%%%%%%%%%%%%%%%%%%%%%%%%%%%%%%%%%%%%%%%%%%%%%%%%%%%
%
%  We now create the contents page and if necessary the list of figures and list of tables.
%
%%%%%%%%%%%%%%%%%%%%%%%%%%%%%%%%%%%%%%%%%%%%%%%%%%%%%%%


\cleardoublepage
\phantomsection
\addcontentsline{toc}{chapter}{Contents}

\tableofcontents
\listoffigures %Use if you have a list of figures.
\listoftables%Use if you have a list of tables.
\lstlistoflistings% Use if you are using the code option

%%%%%%%%%%%%%%%%%%%%%%%%%%%%%%%%%%%%%%%%%%%%%%%%%%%%%%%

%!TEX root = ../username.tex
\chapter*{Preface}\label{pref}
\addcontentsline{toc}{chapter}{Preface}
\lettrine[lines=2, lhang=0.33, loversize=0.1]{T}he  % most theses do not have a preface so this should be commented

%%%%%%%%%%%%%%%%%%%%%%%%%%%%%%%%%%%%%%%%%%%%%%%%%%%%%%%
\mainmatter

%%%%%%%%%%%%%%%%%%%%%%%%%%%%%%%%%%%%%%%%%%%%%%%%%%%%%%%
%
%                                                       Thesis Chapters
%
% This is where the main text of the thesis goes. I have written this template assuming that
% each chapter is a separate file. You do not have to do this but it makes things easier to find
% for editing. You can use the sample chapters to help you figure out how to type things into
% your thesis. To include a chapter just use the \include{chaptername} command. Chapters are
% included in the order listed.
%
%%%%%%%%%%%%%%%%%%%%%%%%%%%%%%%%%%%%%%%%%%%%%%%%%%%%%%%

%!TEX root = ../username.tex
\chapter{Introduction}
Introduce motivation

Define game based learning and give examples. 

goals 

%!TEX root = ../username.tex
\chapter{Game-Based Learning}
Game-based learning is a subset of serious games. Serious games are:
	\begin{quote}
		"played with a computer in accordance with specific rules and produced, marketed, or used for purposes other than pure entertainment; these include, but are not limited to, educational computer games (edutainment) and advertainment" \cite{Bontchev2015}.
	\end{quote}
 Advertainment being any kind of entertainment made with the goal of advertisement. Game-based learning uses entertainment to capture peoples' attention teaching them in a fun and engaging way. There are many advantages to using game-based learning such as increasing the users motor and spatial skills, training, and treating mental illness. 

\section{Applications}  
There are many applications of serious games and game-based learning. Serious games are used in the military and in the government for training. Serious games can be used to train personal by simulating a multitude of environments. Thus providing exposure before entering a new situation. Simulations allow the military and government to train effectively with a significant decrease in cost compared to traditional methods. Serious games can also be created for educational purposes; the goal of these educational games is to train the player in a discipline. Surgeons, for example, can use a game to practice a risky surgery for a specific patient before the surgery is actually performed. Games made to promote health care may include games like "Dance Dance Revolution", where this kind of physical fitness game is referred to as "exergaming". Serious games can be used to diagnose and treat mental illnesses, for example, games can be used in distraction therapy and exposure therapy \cite{Susi2007}. Distraction therapy is a way of helping someone cope with a difficult situation by keeping their attention elsewhere. Exposure therapy is a psychological treatment that helps people confront their fears. This is done by gradually introducing the patient to their fears in a safe environment \cite{Susi2007}. Games can also be used to help people recover and rehabilitate from accidents. Cognitive functioning can also be increased with the use of games. This includes activities like memory training and development of analytical and strategic skill \cite{Susi2007}. 
\begin{figure}[!ht]
	\begin{center}
		\woopic{3edGradeGame}{.6}
	\end{center}
	\caption{A screen shot of the Trials of the Acropolis \cite{Chintiadis}} \label{fig:3edGradeGame}
\end{figure}Games can be used to teach students in subjects such as spelling, reading, and history. For example, one game, titled Trials of the Acropolis, can be used to teach about history using the virtual reality platform. It's goal is to expose students, specifically third graders, to ancient Greek mythology. The game consists of many levels, each of which focus on a different myths from mythology such as; Hercules, Odyssey, and Gods and Titans. A screen shot of the game is shown in Figure~\ref{fig:3edGradeGame}. Each level of the game consists of many scenarios that involve the player solving puzzles and taking quizzes. These puzzles and quizzes help the student test and use what they have learned from the specific level of the game \cite{Chintiadis}. 


\section{Why it Works}
There are four main reasons why game-based learning is effective. The first is the amount of time the learner spends on the material. It has been shown that students tend to spend more time with the assigned material when it is in the form of a game rather than when it is provided in other traditional representations. Second, specific neural receptors in the brain are developed while playing games. These newly developed receptors allow for the student to have better retention of the material. Third, students become more aware of differences in their surrounding stimuli. This allows them to recognize more subtle differences within their environment. Lastly, students tend to be capable of reacting to multiple stimuli at once \cite{LaValle2017}. 


A student spends more time with the assigned material while playing an educational game because they are, ideally, interested and engaged in it. This engagement allows the player to gain a deeper understanding of the material and achieve more meaningful learning and retention. This engagement can also lead to more interaction between the student and instructor, which can result in a higher understanding. There are many arguments as to why game-based learning works. Edgar Dale's Cone of Experience, shown in  Figure ~\ref{fig:whatpeoplerememerfigure}, can be used to explain the effectiveness of game-based learning \cite{Davis2015}.
\begin{figure}[!ht]
	\begin{center}
		\woopic{whatpeoplerememerfigure}{.6}
	\end{center}
	\caption{Edgar Dale's Cone of Experience \cite{Davis2015}} \label{fig:whatpeoplerememerfigure}
\end{figure}The Cone of Experience shows that people are more likely to remember 90\% of what they do, as opposed to 10\% or 20\% of what they read or hear, respectively. This means that when a person experiences something, they are more likely to retain information from that experience. Games, especially games in VR, allow people to have experiences they would otherwise not have. Section~\ref{guidlines} goes into more detail on how this experience can be created and be made effective.   


There have been many studies that test if using games, along with traditional teaching methods, actually improve student's scores. One study examines two groups of students enrolled in an engineering class called 'Materials and Methods of Construction' \cite{Tobias2014}. Both groups take the same exam but are given different ways to study the material. The control group is given the assigned reading while the experimental group is given the assigned reading and instructed to play SimCity 2000, a city construction game. Both groups were given a 20 question multiple choice and true/false exam. After the exam they were given a follow up survey. The experimental group out performed the control group in the exam and based on the follow up survey they enjoyed the computer simulation more than the reading. This shows that between a group of students who are given a game to learn the material versus other methods, the students who use a game tend to do better and have a more enjoyable experience. This may be due to the fact that they spend more time with the material because they enjoy the game or because of the material in the game itself. Either way, the students who used the game outperformed the students who did not \cite{Tobias2014}. 


In another study two kindergarten classes, children from ages 5-6, are analyzed. Most of these students are African-American, of low socioeconomic status, and have one parent. The experimental class, consists of 24 students while the control class consists of 23. The study is testing the students' test results in the subjects; spelling, math, and reading. The experimental group is given educational game consoles to use in class for 11 weeks. The control class did not receive the console but instead hold regular classes. Both the control and the experimental group improve in the 11 weeks however the experiential group shows significantly greater improvement in spelling and reading compared to the control group, but not in math \cite{Tobias2014}. This experiment shows how the use of games in classrooms can be beneficial to a students' learning. 

\section{Guidelines to Creating an Educational Game} \label{guidlines}
When creating an education game it is imperative to make the experience as realistic as possible. The goal of the game should be to activate the part of the player's brain that will be used to perform the learned activity \cite{Tobias2014}. In other words, the part of the brain that activates while playing the game should be the same part of the brain that activates while doing the intended activity. For example, if the game is teaching someone how to hammer a nail, while in the game a part of the players' brain should activate, then when that player goes to hammer a real nail the same part of the brain should activate. To create an experience that activates the correct parts of the brain, immersion is used. Immersion allows the player to be entirely focused and surrounded by the game, completely separating the player from the real world. There are many ways to create an immersive game, one way is to use realistic physics and human-like voices, instead of synthetic voices. Another way is to develop the game in a first-person perspective, allowing the player to experience the world through the eyes of the character. Including dialogue is another important aspect to include, making the player feel as if they are inside the game. The use of characters can be beneficial in creating immersion, these characters should be animated and able to interact with the player. The characters, or objects, can be used by the developers to help guide the players though the game without breaking the player's feeling of immersion. According to one study, these characters should not be aggressive but rather should act socially with the player as this is conducive to a good learning environment \cite{Tobias2014}.


Aesthetics are important in creating an immersive environment. One study asked experts in game design, computer science, and interactive media to fill out a survey. This survey consisted of questions regarding what game aesthetics they thought are most important for a students' perceived learning. Table~\ref{tbl:AestheticsPerceivedLearning} displays the results of experts reviews based on frequency of responses. 
\begin{table}[!ht]
	\begin{center}
		\caption{Game Aesthetics for Perceived Learning \cite{AbuBakar2017}\label{tbl:AestheticsPerceivedLearning}}
		\begin{tabular}{|l|c|c|c|}
			\hline 
			Game Aesthetic & Very Important & Somewhat Important & Not Important  \\ 
			\hline 
			Image & 6 & - & - \\ 
			\hline 
			Text & 5 & 1 & - \\ 
			\hline 
			Visual Perspective & 5 & 1 & - \\ 
			\hline 
			Color & 5 & 1 & - \\ 
			\hline 
			Graphic & 5 & 1 & - \\ 
			\hline 
			Layout & 5 & 1 & - \\ 
			\hline 
			Sound Effect & 3 & 3 & - \\ 
			\hline 
			Voice & 4 & 1 & 1 \\ 
			\hline 
			Music & 3 & 2 & 1 \\ 
			\hline 
			Shape & 3 & 2 & 1 \\ 
			\hline 
			Form & 3 & 2 & 1 \\ 
			\hline 
			Texture & 3 & 2 & 1 \\ 
			\hline 
		\end{tabular}
	\end{center}
\end{table}The experts showed that in their opinion images are most important for learning in a game. Thus images  should be a primary focus when developing an educational game \cite{AbuBakar2017}. 


The game should keep the player interested and motivated to continue to play \cite{Tobias2014}. Two ways of motivating the player to want to continue playing include intrinsic and extrinsic rewards. "Intrinsic motivation pushes us to act freely, on our own, for the sake of it; extrinsic motivation pulls us to act due to factors that are external to the activity itself, like reward or threat" \cite{Cataldo}. To create these rewards the game should include "a system of rewards and goals which motivate players, a narrative context which situates activity and establishes rules of engagement, learning content that is relevant to the narrative plot, and interactive cues that prompt learning and provide feedback" \cite{Cataldo}. Games that invoke deep learning and understanding rely on motivating the player to use higher order thinking rather than rote memorization or simple comprehension \cite{Cataldo}. 


Developers should be aware of the players cognitive load while in game and attempt to reduce it as much as possible. The cognitive load is the amount of effort someone must put into their working memory. Working memory is a part of short-term memory. By reducing the amount of effort required to think about the details of the game, the learner is able to place more energy in linking the new concepts to their prior knowledge. Figure~\ref{fig:MemoryandLearning} shows how information works it's way through the brain and into long term memory by way of the working memory. Reducing cognitive load can be accomplished by making the game flow naturally without the player having to struggle through it, making the game intuitive and realistic. 
\begin{figure}[!ht]
	\begin{center}
		\woopic{MemoryandLearning}{.7}
	\end{center}
	\caption{The process of information being processed and retained \cite{Tobias2014}} \label{fig:MemoryandLearning}
\end{figure} To help the player move through the game easily is it important to give them guidance. As previously stated, characters and objects can be used to provide the user with instruction and information. Pictures and demonstration, observing how to do an intended activity, are good methods of showing a player how to accomplish an intended activity. After the user is successful in completing the activity, it is important to give them positive feedback and encourage them to reflect on the correct response. The game should provide guidance that encourages the player to learn through discovery, allowing them to find the answer on their own rather than directly telling them what they are to do \cite{Tobias2014}. 


Immersion is not the only aspect that is important in creating a successful educational game. The game should be extremely engaging and allow the player to investigate the world. The player should be involved and invested in the game. This can be done by allowing the player to interact with the world (objects, characters, etc.), involve them in an interesting and immersive storyline, and empathetic characters \cite{Tobias2014}. More of how to do this is discussed in Chapter~\ref{gamedesign}.


While developing an education game it is important to work in teams as this allows for more perspectives and a more widely effective end result. Testing the game during development is integral in determining the effectiveness of the game in it's intended results. One way to do this is have users take a small test before playing the game. After they play the game for a set amount of time they should be tested again. If their test score had a significant change then the game was successful, if not then revisions to the game and the test should be done. After the user plays the game it is also important to ask the user their feelings on the game and what they felt was effective and what was not. It is also possible to have the user play the game while connected to a brain monitor allowing the experimenters to watch the players' brain activity. After recording their brain activity the experimenters can have the users perform the newly learned task with the game brain monitoring device. If in both the game and the activity, the same part of the brain was activated then the game was well designed. Tf not, the game should be revised and the test should be done again. It is also important to be aware of the emerging research findings. These tests should help the developers revise the game to make the game more effective, as stated earlier, the test and revise cycle should continue until the intended results are reached \cite{Tobias2014}.   

%flow psychology

\section{Conclusion}
Game-based learning can be a useful tool in teaching and learning. It allows a student to become more invested in the material and retain more ofsss information. It is possible to create a good game for education by making it an immerse and realistic experience.  
%stregthen this 
%!TEX root = ../username.tex
\chapter{Game Design} \label{gamedesign}
Many components are needed to create a game that is engaging and interesting. The game world should be consistent and give the player a sense of power. To create an interesting narrative, it is important for the story to have non-linear game play. This can be done by creating multiple storylines where the player's path is based on their actions and choices thus allowing for multiple outcomes. Another way to create non-linear game play is to allow multiple solutions to a singular problem. This allows players to find their own ways to complete a task and use their own creativity while doing so. This can also be done by giving the player the freedom of choosing the order in which tasks are completed. Lastly, a good game allows the the player to pick and choose which tasks they want to complete. Meaning all the game's tasks are not mandatory. This gives the player the ability to choose to only complete the tasks they find interesting while not having to be burdened with tasks they may find tedious or boring. By allowing the player to choose the tasks they complete the game will keep their interest longer \cite{Iii}. 


Structure can be given to the player through the use of rules and win states. While playing the game, users are encouraged to continue if they are given feedback. Feedback is like reinforcement, there are two kinds of reinforcement, positive and negative. Positive reinforcement is where the player is given a desired element. For example the acquiring of points, coins, ammo, or food would be positive reinforcement. Negative reinforcement is when an undesirable element is released from the player. For example, the loss of an ailment, an enemy, or a barrier would be considered negative reinforcement \cite{Skinner2014}. It is also important to allow the player to problem solve and find their own unique solutions. The game should spark creativity and invoke emotion in the player \cite{Nguyen2012}. 


A game should allow the player to have creative and destructive powers. Creative powers allow the player to bring something into existence, destructive powers allow the player to alter the environment. Giving a player these powers allow them to feel empowered and give them to have a sense of satisfaction and creativity \cite{Behavior2009}.  


An engaging game has challenges, some standard challenges include; time challenges, dexterity challenges, endurance challenges, memory challenges, logic challenges, and resource control challenges. These challenges allow for the player to become engaged and invested in the story. However, challenges should not be placed back to back with each other, allowing the player to recuperate from a challenge is important in keeping their endurance. The game's story must possess a clear beginning, middle, and finale, giving the player a sense of closure \cite{Nguyen2012}. 

\section{Storyboard}
One of the first processes in game design is creating a storyboard. A storyboard is a set of drawings that represent the different scenes in the game. The same for this project, it begins with the player entering an excavation site. Their supervisor explains how excavation is done and shows them the layout of the grounds. There is a tent where the player rests when they become fatigued. Once artifacts are found, they are transported by the jeep them to the lab where they are analyzed. After the team at the lab has analyzed the materials they update the computer records with the new  information about the artifact. A picture storyboard is located in the Appendix.                                     
%\begin{figure}
%	\centering
%	\includegraphics[width=0.7\linewidth]{../../StoryBoard/StoryBoard_Beginning}
%	\caption{}
%	\label{fig:storyboardbeginning}
%\end{figure}


\section{User Interface}
The user interface is an important aspect to any virtual reality game. Without an interface the user does not know what to do or how to do it. An interface allows the player to interact with the game. There are many kinds of user interfaces, each having their own advantages and disadvantages. Two commonly used interfaces are called diegetic and non-diegetic. A diegetic interface involves the interface as apart of the virtual world; a working clock for example. Non-diegetic interfaces involve information that is stationary. 2D image or text that sits on top of the screen; a health bar for example \cite{Iacovides2015}. Non-diegetic interfaces are rarely used in VR due to the users difficulty of focusing on an object that is constantly so close. Spacial interfaces are VR's version of non-diegetic interfaces and are discussed in further detail later in this section.   
\begin{figure}[!ht]
	\begin{minipage}[!ht]{6cm}
		\woopic{diagetic}{.9}
		\par
		\caption{Diegetic Interface \cite{Iacovides2015}.}
		\label{fig:Diegetic}
	\end{minipage}
	\hfill
	\begin{minipage}[!ht]{6cm}
		\woopic{non-diagetic}{.9}
		\par
		\caption{Non-diegetic Interface \cite{Iacovides2015}.}
		\label{fig:Non-Diegetic}
	\end{minipage}
\end{figure}Figure~\ref{fig:Diegetic} shows Battlefield 3 with a diegetic interface and Figure~\ref{fig:Non-Diegetic} shows Battlefield 3 with a non-diegetic interface. Figure~\ref{fig:Diegetic} is diegetic because it does not have 2D images on the screen like Figure~\ref{fig:Non-Diegetic} does. In Figure~\ref{fig:Diegetic}, the bottom right and left hand corners do not have boxes of information and there are not 2D indicators around the world displaying information. as Figure~\ref{fig:Non-Diegetic} is non-diegetic because it does have those 2D displays of information. In one study, subjects experienced in gaming played the same game, Battlefield 3, two times. One time they played it with a diegetic interface, shown in Figure~\ref{fig:Diegetic} and once with a non-diegetic interface, shown in Figure~\ref{fig:Non-Diegetic}. This study found the subjects felt more immersed and engaged in with the diegetic interface. One of the subjects stated "I preferred the challenge of the first game...everything was kind of hidden as well so I had to figure it out for myself. So I seemed to get more involved in it I think" \cite{Iacovides2015} and another said "I did feel that I felt a bit more immersed in the version without the interface simply because there were no flashy things around. So compared to the other one where I was constantly looking at the screen and darting between everything, so I was able to focus on the actual gameplay more". \cite{Iacovides2015} The experimenters then performed another study in which novices and experienced players played the same game in diegetic and non-diegetic forms. The novice players tended to enjoy the non-diegetic interface better because it gave them more information and instruction on how to play. The experienced players did not share this frustration because they were already aware of how to play. Perhaps a non-diegetic interface should have the option to be removed after a user has had adequate experience in playing the game. The removal of non-diegetic interfaces allows for a more immersive and cognitively challenging experience for experienced users \cite{Iacovides2015}. 


Spatial user interfaces are another commonly used interface in virtual reality worlds. It is a good combination of non-diegetic and diegetic interfaces. Spatial interfaces consist of information that is placed inside the world, for example a 3D number floating above the users gun could indicate the amount of ammunition left in the gun. The floating number would be attached to the object so it would move with the gun as the user interacts with it. An example of this can be seen in Figure~\ref{fig:spatialUI}.
\begin{figure}[!ht]
	\begin{center}
		\woopic{spatialUI}{.6}
	\end{center}
	\caption{An example of spacial user interface in VR, photo from unity3d.com/learn/tutorials/topics/virtual-reality/user-interfaces-vr} \label{fig:spatialUI}
\end{figure}
Spacial user interfaces became necessary in the 1990s. This is because as computer tasks became more complex, users began needing a multitude of windows open on the screen. At the time, most user interfaces were non-diegetic and 2D. Ways to combat the cluttered 2D windows on the screen began with the use of task bars and a "window list". These were helpful in organizing the windows but it was still not enough. This is when spatial interfaces began to come into existence. One of the first 3D interfaces was patented in 2006. It consisted of a sphere that is lined with icons linked to different windows. Because of it's 3D shape, not all windows can be viewed at once but it allows for the windows to be located in a small 3D sphere instead of all over the screen. One disadvantage to the sphere is that the user may have problems deciding a logical placement for all of the icons on the sphere \cite{Roca2006}. Spatial interfaces allow for more information and instruction to be given to the user without cluttering the screen and breaking their immersion as much as non-diegetic interfaces.

\section{Conclusion}



%!TEX root = ../username.tex
\chapter{Archaeology}
Archaeology is the study of humans throughout time. Archaeologist learn about the people of the past by discovering and analyzing objects they left behind. This study focuses on period of the Neanderthals. This area of study is often refereed to as the study of pre-history because most objects found during this time predate modern humans. This project is strongly influenced by an archaeological site called Cova Gran de Santa Linya. Cova Gran is a rock shelter located in Lleida, Spain. Archaeological evidence found at the site has been from 50,000 years ago.    

\section{Who were the Neanderthals}
Neanderthals (the 'th' pronounced as 't') began appearing roughly 200,000 to 100,000 years ago. They continued to exist until about 30,000 years ago when they vanished. Modern humans appeared 40,000 years ago, this means there was approximately 10,000 year overlap between modern humans and Neanderthals. There is much speculation as so why Neanderthals did not survive while modern humans flourished and whether or not Neanderthals and modern humans had any interactions \cite{Smithsonian}.


Neanderthals lived in groups and took care of each other. This is evident from Neanderthal skeletons that are found with bones that were broken and healed within the Neanderthals lifetime. Skeletons have also shown signs of the Neanderthal being toothless for a significant period of time before death. These examples imply that during recovery or before death, these Neanderthals would have needed assistance to live. 


There is also evidence that Neanderthals were capable of speech and the use of symbolism. Ochre and manganese, "crayons", have been found at Neanderthal sites implying their use \cite{Harvati2010}. There is some evidence that Neanderthals may have buried their dead, although it is in no way proven. A Neanderthal skeleton has been found with flowers around it. It is unclear if these flowers were deliberately placed or blown by wind. It is thought that there is a high mortality rate among you and prime age Neanderthals due to quantity and age of found remains.    
 

There are many theories as to why the Neanderthals did not survive. One hypothesis is that Neanderthals assimilated with modern humans. The Neanderthals could have been integrated, assimilated, into modern human groups. They also could have lived separately still sharing technology or trading with each other. It is possible that Neanderthals and modern humans met, worked, or lived together. Another hypothesis is Neanderthals became extinct due to indirect competition with modern humans. Modern humans' advantages may have included larger group sizes, slightly higher birth rates, lower mortality rates, shorter interbirth spacing, greater dietary diversity, more complex social networks, and better clothing and shelter \cite{Harvati2010}. 


\subsection{Appearance}
Today there is a common misconception about the Neanderthal appearance. In 1909 the common belief was that Neanderthals were stupid and crude in comparison to modern human ancestors. 
\begin{figure}
	\centering
	\begin{minipage}{.5\textwidth}
		\centering
		\woopic{1909}{.83}
		\caption{A depiction of Neanderthals from 1909 \cite{larsen_2017}.} \label{fig:Neanderthal-1909}
	\end{minipage}%
	\begin{minipage}{.5\textwidth}
		\centering
		\woopic{neanderthal_Recon_Head}{.26}
		\caption{A reconstruction of a Neanderthal head from present time \cite{Smithsonian}.} \label{fig:Neanderthal-recon}
	\end{minipage}
\end{figure}They were believed to be covered in hair and appear more ape-like than human, shown in Figure~\ref{fig:Neanderthal-1909}. In media today Neanderthals are depicted as brutish cave men however this is an outdated portrayal. Archaeologist no longer believe this description of Neanderthals to be true. Neanderthals are now thought to have appeared more similarly to humans than apes and have a comparable brain size \cite{larsen_2017}.


Neanderthal bodies were designed to withstand the cold. They had large nasal openings which allowed for more air to be heated before entering the lungs. The brain size of Neanderthals is larger than most modern humans. A reconstruction of a Neanderthal head is shown in Figure~\ref{fig:Neanderthal-recon}. They had relatively short limbs and a short stature. 
Neanderthal males were an average height of 5 ft 5 in, females were an average height of 5 ft 1 in. The average weight for male Neanderthals was 143 lbs, the average weight for a female was 119 lbs \cite{Smithsonian}. The short stature and short limbs are classic adaptations to the cold. Less skin surface area means less energy to heat. 


\subsection{Tools}
The Neanderthals used many types of tools to help them achieve everyday tasks. To protect themselves from the elements Neanderthals made clothing and lived in shelters. They would create tools for sewing such as needles and awls. They used fire for its warmth, protection from predators, cooking, light, tool tempering, vegetation clearing, hunting, preserving food, and pest control. The size, shape, and placement of an excavated hearth, floor of a fire place, can help explain what the fire's use was. Hearths that are found next to the rock walls may have been used to warm the rock and help heat a sleeping area. Hearths that have holes on either side of them, presumably where sticks were placed vertically, may have been used for cooking. Some hearths have been found with large rocks on them, this may have been done by the Neanderthals to stop the use of the hearth. The inspection hearths, charcoal, can also indicate what kind of wood was burned, what was being cooked, and the temperature of the fire. Hearths that held high temperature fires indicated that the users of the hearth had some kind of understanding of how fires can be created and maintained. This information can help archaeologist speculate as to what they know and techniques the Neanderthals may have used.  


Neanderthals also made and used stone tools. In archaeology a lithic is a stone that has been manipulated by a  human ancestor or a Neanderthal. Tools were used in many activities such as hunting, cleaning hide and cutting wood and food. This is evident by lithic material being found in hide, animal horns, and in wood. Tools were not only created but also specialized. Different activities required different tools. 


Neanderthals hunted in groups and often used planned techniques to hunt successfully. There have been some well-preserved wooden spears dating 400,000 years ago \cite{Harvati2010}. It is uncommon to find wooden artifacts, such as these spears, because wood needs specific circumstances to become preserved. It is likely other wooden tools were made and used by the Neanderthals. It is hypothesized that Neanderthals did not throw these spears but instead would use them with a pushing motion. This is thought to be true because of their upper body bone structure. They would not easily be able to put their arms above their head and would have a stronger force behind the pushing rather than throwing motion. They also lead animals off cliffs or make traps to catch prey. Hunting however was extremely risky and required a lot of energy and calories. Some Neanderthal skeletons have been found having similar injures to that of bull riders. It was more common for Neanderthals to scavenge than to hunt. If they found a carcass it was most likely mostly cleaned by birds with the bones being left behind. To extract nourishment from the bones, Neanderthals would use stones to beak open the bone and eat the marrow from inside.


\subsubsection{Tool Classification}
To make stone tools Neanderthals used a method we have named knapping. Knapping is the action of hitting one rock, called the hammer stone, against another rock. Flit was a common material used to make tools because when hit, sharp flakes break off. Quartzite is a bit harder to knap but easier to find. Limestone has also been used but it very easily broken. While knapping there debris falls off of the flint as well as the intended tool. 
\begin{figure}[!ht]
	\begin{center}
		\woopic{neanderthal-tools}{.26}
	\end{center}
	\caption{Neanderthal toolkit : https://www.amnh.org/exhibitions/permanent-exhibitions/anne-and-bernard-spitzer-hall-of-human-origins/neanderthal-tools. } \label{fig:Neanderthal-tool}
\end{figure}The hammer stone is usually round and made from a material harder than the stone being knapped and is usually discovered intact. Neanderthals could strike the flint in deliberate ways to create different tools. After a flake could then be further developed with retouching. Retouching is when the flake is knapped to create something new. A round flake, as seen in Figure~\ref{fig:Neanderthal-tool}, is called an end scraper and can be used to clean hide. Blades are taller that they are wide, these can be used in hunting. A notched scrapper, shown in Figure ~\ref{fig:Neanderthal-tool}, is used to take the bark off of pieces of wood. It would be used by placing the wood in the notch and scraping. Some stones were made for chopping wood and digging. After a piece of flint is too small to strike again it was usually discarded by the Neanderthals, this discarded piece is called the core. 

\subsection{Diet}
Neanderthal sites have been found with many animal remains indicating meat being apart of their diet. Animals that have been commonly found at Neanderthal sites include bison, wild cattle, horse, reindeer, red and fallow deer, ibex, wild boar, and gazelle. It is even thought that larger animals such as wholly rhinoceros or wholly mammoth were consumed by Neanderthals \cite{Harvati2010}. At some sites there has been remains of shellfish, birds, and marine mammals \cite{Harvati2010}. Neanderthals did eat plants but most of their diet was made up of meat. 


\section{Cova Gran}
Cova Gran is an interesting site because of the period at which artifacts have been dated. Cova Gran has artifacts ranging from before Neanderthals vanished, to the overlapping of Neanderthals and modern humans (40,000 - 30,000 years ago), to after Neanderthals vanished. Cova Gran is a rock shelter, not a cave, meaning it is a rock overhang that can give protection from the elements. It is thought to have been used as a temporary living space for nomads or shepherds throughout the years. Today Cova Gran's opening is much larger than it would have been during the time of the Neanderthals due to the rock ceiling falling gradually over time.    
\begin{figure}[!ht]
	\begin{center}
		\woopic{CovaGrab}{.5}
	\end{center}
	\caption{Photo of the Cova Gran de Santa Liny field site \cite{CovaGran}.} \label{fig:CovaGrab}
\end{figure}  


Every summer a team of archaeologists and students work on excavation at Cova Gran. For an average archaeologist, who's job is to excavate the site, most days consist of excavation, cleaning, analyzing, and organizing artifacts. 

\section{Game Topics}
This projects covers some topics of a Neanderthal and archaeologists life. Players learn about how Neanderthal made ad used tools, what they ate and hypotheses as to why they vanished. Players also learn about what it is like excavating at an archaeological dig site.  
 



%how Neanderthals build a fire I was thinking of how I could implement fire building into the game. 
%First the player would need to gather all the materials needed for the kind of fire making they 
%wanted to use. If they were using the percussion technique they would need to gather the 
%correct type of stones, tinder such as a small bundle of dried grass, and some larger sticks to get 
%burning. Once those materials were collected they could have the action ability to hit the rocks 
%together over the tinder and make sparks. In the code for this bit I could have likelihood that the 
%tinder would catch, a 50 percent chance of the tinder lighting would work well. Then once the fire 
%was lit that activity would be complete. 


\section{Conclusion}


%FROM ESSAY

%The tools used in excavation include a rubber bucket, a screwdriver, a box of thumb pins, multiple small bags, a paint brush, a small shovel, small tubes, a wooden stick, two sitting mats, and labels. 
%Once gathering the tools, I would go to the site and remove the tarp covering the pit and 
%sit in my designated area. The directors decide what areas are to be worked on that day and then
%the work is divided between the people working that day. It is important to keep the entire pit at 
%roughly the same level and not have areas that are significantly higher or lower than the other. 
%Once I know what area I am responsible for that day I set my mats down and lay out my 
%supplies. Before I begin digging I look at the area and decide where I need to start first. In the 
%specific area I was working there was a slope in the level, the bottom of the slope was in the 
%south east corner, it is also important to be in contact and work hand in hand with the people 
%around you so that the slope is consistent between everyone in the pit.  
%Once deciding where I should begin to dig, finding where the slope should be and 
%talking to the people around me I can begin to make the first marks in the soil. I begin by 
%cleaning the area but brushing any debris that has gathered into my trowel and emptying the 
%contents of the trowel into the rubber bucket. Once the area is clean I take the screwdriver and 
%press into the soil at a 45-degree angle which in turn breaks up the soil and loosens the dirt. I 
%work like a typewriter going from left to right working in rows downward until I reach the 
%bottom of my work area. Once a thin layer of soil is loosened I use the large paint brush to 
%gently push the loose dirt into the small trowel. The dirt from the trowel is then placed into the 
%rubber bucket, this activity is referred to as cleaning. If during the cleaning process any piece of 
%artifact is revealed then a push-pin is placed next to the artifact so it is easy to see and not loose 
%track of. The artifact is then dug around to make sure is not damaged and is not picked up until 
%it looks like it is no longer in the ground, this is done to prevent holes being created in the layer. 
%When digging closer to the artifact I used a wooden stick to release it from the dirt as to not 
%damage the artifact.  When the artifact is clear from the ground it is given a label. Each level has different 
%labels, for example when digging in level V-12 each artifact's label will say V-12 on it with the corresponding code for V-12 as well as an individual number and code so it can be identified later on. I then take the coordinate of the object, this is done by using the prism and the total station transit. I tell the person controlling the total station what level I am working on, I then place the tip of the prim next to the artifact and make sure it is vertical using a bubble level that is built into the prism. When the bubble is steady and prism is vertical I yell the number of the 
%artifact and wait for the person controlling the total station to tell me they successfully got the 
%point. It was important for me to clearly say the number of the piece because it is easy to mix 
%things up so many double checks are in place, if I say a number out of turn then the person 
%controlling the total station can correct me and I can find the correct piece as to not mix all the 
%artifacts locations up. After the artifacts location is placed then I put the artifact and its label 
%into a small plastic bag and tie it loosely just s the artifact will not fall out. At Cova Gran 
%artifacts are commonly either lithics, small animal bones, or charcoal. If a material is too small 
%or insignificant to get a label and point to itself it is placed in a no coordinate bag along with 
%others of its kind. When the rubber bucket has a few cleanings worth of soil it is taken to the sift. The 
%content of the rubber bucket is place onto a screen and shaken to be rid of the excess dirt, then 
%the screen is placed into a large bucket of water and shaken to clean the remaining materials. 
%Once the materials are clean it is easier to see what everything is and it is important to look to 
%see if you missed any artifacts. It is common to find no coordinates in the sifting process due to 
%their smaller size. If there is something important found in the sift it is brought back to the site 
%and randomly placed in the area you were working to be given a label and a coordinate.   
%After excavation the lab work begins. Lab work entails cleaning, labeling and categorizing. In the field the materials for each level are placed in a larger "mother" bag this bag is then transported back to the lab to begin the next steps of the process. The first thing that is done to the materials in the lab is cleaning, to clean materials a shallow tub of water is placed in the middle of the table. While cleaning it is important to keep the materials from different levels separate as to cause less trouble down the road. 


%The first step of cleaning is opening all the bags, then all the materials and their corresponding labels are placed on the table in row. Then with one hand the material is picked up and with the other hand it is placed in the water and 
%rubbed with the fingers to remove any dirt or debris. To place the material back on the table the 
%other hand is used to decrease the amount of water on the material and table which in turn 
%decreases drying time. This is done to all the materials in the rows for that level, once they are 
%done then the no coordinates can be cleaned. No coordinates are cleaned by being placed in a 
%small sift and placed into the water, if any larger materials are inside then they can be cleaned 
%by hand. Charcoals are kept in their tubes and not cleaned, they are placed in a separate area 
%with their caps off allowing them to "breath". There are some pieces that should not be cleaned 
%because they are either to fragile and would fall apart in the cleaning process or they would lose 
%some information that is on the exterior of the material. For example, hammer stones are not 
%washed because they can be analyzed to see what is on them which can lead to a better 
%understanding for what it was used for.  


%Once all the materials are dry it is time for the labeling processes. To label the materials 
%you need a pair or tweezers, clear nail polish, and a wooden stick. On the labels that have been 
%put with the materials there are three sizes of QR codes. These QR codes are linked to a number 
%in the database which stores all the data about the specific material. To access this information 
%all that needs to be done is scan the QR code and the information for the material is accessed 
%and displayed this information can be added to and edited if need be. The first numbers linked 
%to the QR codes show that the material was found at Cova Gran, the next numbers correspond 
%to the level the material was found, and the last numbers correspond to the specific number 
%given to the material. The smaller codes only have the last bit of information, so it is better to 
%use the larger codes when possible. To place a QR code onto an artifact the first step is to 
%decide where the code should go, if it is a lithic it should be placed on the ventral side as far 
%from the edge as possible. The goal of placing QR codes is to link the material to the database 
%and not cover up important information about the piece such as burnt areas or a place where the 
%piece was retouched. It is also important to place the code on a clean, smooth, and flat surface 
%to prevent the code from falling off. In some cases, a wooden stick can be used to scrape away 
%any concretion that is in the way of optimal placement of the QR code. When the proper 
%placement is found a dab of the clear nail polish is placed in that location. It is important to 
%make sure that the nail polish is not too thick because if that is the case it is probable that it will 
%dry bubbly and the code would not scan and the material would need to have a new code placed 
%on it. After the first coat of nail polish is on then with tweezers the QR code is chosen and 
%placed in that area. It is important to make sure the entire code was printed on the sticker, so it 
%will be able to be scanned later on. Then a second layer of nail polish is placed on top of the 
%sticker with the code on it to keep it in place. Once all the materials in the level are marked with 
%their code they are placed in a shallow box and each one is scanned to make sure the QR code 
%works and shows the correct number. After this is done they are ready to be classified.  
%To begin the classification, process a single level is chosen, this level will be completed 
%before moving to a different level to avoid mixing the levels. Smaller boxes are gathered and 
%are designated to hold one type of material. 





%!TEX root = ../username.tex
\chapter{Virtual Reality}
Virtual reality (VR) is continually evolving and as new developments are made the definition changes and expands. The core concept of VR can be described as follows:  
	\begin{quote}
	"Inducing targeted behavior in an organism by using artificial sensory stimulation, while the organism has little or no awareness of the interference" \cite{LaValle2017}.
	\end{quote} 
The "targeted behavior" is the experience created by the developer, for example walking or flying. The "organism" can be any living thing, for example a human, fish, or monkey. The "artificial sensory stimulation" is the replacement of one or more of the organism's senses with an artificial stimulation. For example, sight or sound can be stimulated by a display or headphones. Lastly, "awareness" implies the organism's feeling of presence in the virtual world, allowing the organism to accept the virtual environment as being natural \cite{LaValle2017}. 
\label{key}
There are two main ways to construct a virtual environment. The developer can create a synthetic or captured world. Synthetic worlds are completely generated from geometric primitives and simulated physics, in other words it is completely computer generated. Captured virtual worlds are created from the real world by using modern imaging technologies. Captured worlds tend to lose information from the original environment and are harder to interact with while in VR than synthetic worlds \cite{LaValle2017}.

One crucial component to VR is interactivity, the dependency of sensory stimulation on an action taken by the organism. If there is no interaction in the VR system then it is considered to be an open-loop, if there is interaction then it is a closed-loop. In the case of an closed-loop VR system the organism has partial control over the simulation. This could entail body motions such as movement of the eyes, head, or hands \cite{LaValle2017}. This control allows the user to interact and explore the environment creating a more immersive experience. 

\section{History of Virtual Reality}
Virtual reality technology is still in development. The simplistic and least immersive style of VR is known as desktop VR, also called Window on World (WoW). In desktop VR a monitor is used to display the world and no other sensory output is supported. Fish tank VR is slightly more complex and slightly more immersive by supporting head tracking technology in this case, when the user moves their head the displayed image "moves" accordingly, this however usually does not support other sensory output. Finally immersive systems such as the Oculus Rift allow the user to be totally immersed in the world with the use of head mounted displays and support of other sensory output such as audio and haptic feedback \cite{Mazuryk}.


\begin{figure}[!ht]
	\begin{center}
		\woopic{FirstHMD}{.75}
	\end{center}
	\caption{The first head mounted display created by Ivan Sutherland in 1968 \cite{Mazuryk}. } \label{fig:FirstHMD}
\end{figure}Ivan Sutherland had an interest in creating a virtual world that looked, felt, sounded, and responded realistically as if it was the real world. In 1965 he created the Ultimate Display that included interactive graphics, force-feedback, sound, smell, and taste. He however did not stop there. In 1968 he created the first head mounted display (HMD) with head tracking and an accordingly updated display. This is shown in Figure~\ref{fig:FirstHMD}. Although the created images were not realistic and only simple solid 3D objects, this development marked the beginning of modern virtual reality technology \cite{GutierrezA.2008}. After this in 1971 the "GROPE" was created. It was the first prototype of a force-feedback system and was created at the University of North Carolina. It used a previously developed haptic robot called the ARM (Argonne Remote Manipulator) that received feedback signals from sensors mounted to the robot that applied forces to the user \cite{GutierrezA.2008}. After the creation of the first HMD, development for VR began to increase. 
\begin{figure}[!ht]
	\begin{center}
		\woopic{WindTunnelHMD}{.75}
	\end{center}
	\caption{NASA Ames Virtual Wind Tunnel: (a) outside view, (b) inside view \cite{Mazuryk}. } \label{fig:WindTunnelHMD}
\end{figure} VCASS, developed in 1982 by the US Air Force's Armstrong Medical Research Laboratories, was an advanced flight simulator that allowed the user to view graphics showing targeting or optimal flight path information. Other HMD's were being developed, for example the VIVED, Virtual Visual Environment Display, was created by NASA Ames in 1984, and in the 1990's they also developed a Virtual Wind Tunnel. Some of the early head mounted displays and NASA Ames Virtual Wind Tunnel are shown in Figure~\ref{fig:WindTunnelHMD}. Advancements for the HMD continued and are being improved upon today. 


\section{Applications}
Virtual reality can be used not only for entertainment but can also be valuable in other fields. VR can be used to virtually transport people to places they would not otherwise be able to visit, for example visiting an ancient roman temple. Communication can be improved by the use of telepresence, using VR to interact and converse with people from around the world without leaving the room. VR can be used in education, allowing students to visualize complex concepts or data and can be used to create simulators for practical training. In health care, VR can be used to train doctors remotely and be used to create an interactive 3D model of anatomy. Doctors can even perform a virtual surgery to perfect their skills. Models of human muscles, bones, and skin can also be created to help people learn about their interaction. Figure~\ref{fig:HumanModel} shows some 3D models of the human body created for VR that can be interacted with.  
\begin{figure}[!ht]
	\begin{center}
		\woopic{HumanModel}{.75}
	\end{center}
	\caption{In interactive model of the human muscles (Left) and a representation of the stress on the hip joint (Right) \cite{GutierrezA.2008}.} \label{fig:HumanModel}
\end{figure}In therapy, VR can be used to help patients overcome phobias and stress disorders though repeated exposure, improve or maintain cognitive skills, and improve motor skills to overcome balance, muscular, or nervous system disorders \cite{LaValle2017}. VR can be used to help people who suffer from phantom limb after having had their limb amputated. For example someone could visualize their arm as really there with a VR headset while learning to use a prosthetic limb. VR is also a valuable resource when prototyping, designers can visualize their designs before creating them thus saving time and money \cite{LaValle2017}. 
%Add a specific case 

\section{Human Biology and Virtual Reality}

To stimulate an organism's sensory system it is important to understand how it's sensory system works. In this case, humans are the target organisms. 70\% of the human experience is made from sight stimuli, 20\% is from auditory stimuli, 5\% from olfactory stimuli, 4\% from physical stimuli, and 1\% from flavor stimuli \cite{Mazuryk}. With today's technology, VR is focused on the stimulation of the human sense of sight, then sound, and finally touch. 


Each sense has a corresponding receptor, these receptors respond to stimulus, a stimulus can be a sight, sound, feeling etc. When the human body detects a stimuli, the corresponding receptors activate. These receptors then fire certain neural impulses. 
\begin{figure}[!ht]
	\begin{center}
		\woopic{HumanSences}{.75}
	\end{center}
	\caption{A classification of the human body senses \cite{LaValle2017}.} \label{fig:HumanSences}
\end{figure}These impulses translate to one of the senses. The human body has many types of receptors and each responds to different stimuli. The receptors in turn activate different neural impulses. Figure ~\ref{fig:HumanSences} indicates the connection between the human senses, stimulation of the senses, receptors used in the neural transmission, and the sense organ. Specifically, within the human eye there are over 100 million photoreceptors, which perceive visible light. The human body uses mechanoreceptors to sense motion, vibration, or gravitational force. Thermoreceptors detect changes in temperature and finally, chemoreceptors provide neural signals based on the chemical composition of a substance that is in contact with the nose or mouth. In developing stimuli that makes these receptors fire in the brain, sensation is created. These sensations lead  to the player feeling as if the virtual world is as authentic as reality.

\subsection{Seeing in Virtual Reality}
In Virtual Reality it is important to understand the biology of the human brain and eye in order to create a virtual world that can be perceived and understood by the user. Humans have two main ways to use their eyes to gain information about their environment, binocular cues, using both eyes, and monocular cues, using one eye. The main function of these cues is to develop a sense of depth perception. This section provides more detail about how the eye perceives the world and how this information can then be applied to create a virtual environment.


\subsubsection{Binocular Cues} \label{sec:Binocular Cues}
Binocular cues involve using both eyes to understand an image. 
\begin{figure}[!ht]
	\begin{center}
		\woopic{EyeConvergence}{.5}
	\end{center}
	\caption{This figure shows the position of human eyes when focused on near and far objects \cite{JIS2}.} \label{fig:eyeconvergence}
\end{figure} The physical placement of the eyes and the image each eye sees are used by the brain to understand the whole image, where objects are placed and their size. Within a virtual environment the user should be able to view a three-dimensional (3D) virtual world. For this to occur the brain must be convinced that it is in a 3D world instead of looking at a 2D screen. 


To create the 3D environment the illusion of depth is created. There are many methods to accomplish the aforementioned. While focusing on an object that is further in the distance, the viewer's pupils rotate away from each other. This is known as divergence. When an object is closer to the viewer, the pupils rotate towards each other to focus on the object. This is known as convergence. An example of how the eyes are positioned when looking at objects at different distances is shown in Figure~\ref{fig:eyeconvergence}. The brain uses the position and angle of the eyes to determine the distance of an object. In the use of head mounted displays the user views a display that is only inches away from the eyes. The short distance between the screen and eyes makes it impossible for the user to judge an on-screen object's distance based on the convergence or divergence of their eyes due to their eyes being focused on the screen. However, there are other methods to perceive depth of field. Each eye has a slightly different view of the world. The brain can compare the placement of an object within these two images and determine the depth of the object. This is known as stereopsis \cite{JIS2}. On head mounted displays each eye is shown a slightly different image. The slightly different images allow for the brain to perceive depth when comparing the two images. 
\begin{figure}[!ht]
	\begin{center}
		\woopic{LeftRightEye}{1.1}
	\end{center}
	\caption{The images displayed for the left and right eyes on a head mounted display \cite{LaValle2017}.} \label{fig:LeftRightEye}
\end{figure} Figure~\ref{fig:LeftRightEye} shows the difference in images for the right and left eye's display in a head mounted display. 


\subsubsection{Monocular Cues}
Binocular vision is not the only way the brain can perceive depth. Monocular vision involves the use of one eye to focus and interpret an object. Monocular cues are extremely helpful to use in VR to create a more realistic environment.   


Imagine you are stationary and looking into a snow storm. You focus on a a snow flake that is falling about a foot in front of you and it seems to be moving quickly, entering and exiting your field of view in a short amount of time. Then, you focus on a snow flake that is falling about 20 feet in front of you. It seems to be moving more slowly, entering and exiting your field of view in a longer amount of time. Although we know they are moving at the same speed it seems the one further away is slower. This is referred to as motion parallax. Motion parallax can be utilized in VR to create depth by allowing game objects closer to the user to move faster in comparison to the same object that is located further in the distance \cite{JIS2}. 


Now imagine we are in a forest where we can see a cabin. We can tell that there are trees behind the cabin because the cabin blocks the view of part of the trees. This is known as occlusion and can be used in VR games to create a sense of depth. Figure~\ref{fig:Perspective} shows an example of occlusion. 
\begin{figure}[!ht]
	\begin{center}
		\woopic{Perspective}{.4}
	\end{center}
	\caption{Occlusion and linear perspective \cite{JIS3}.} \label{fig:Perspective}
\end{figure} 


In our wooded scene we can demonstrate relative size. We know in reality most of the trees are about the same size but some of them look smaller in size because they are further in the distance. We know this because of relative size, we perceive the trees that look smaller as being farther away, and the trees that look larger as being closer \cite{JIS2}. We can use this in VR to simulate depth by making similar sized objects at different distances different sizes.


Now imagine there is a bear running towards us in the woods. As the bear gets closer to us it seems to be growing in size. We know the bear did not actually grow in size within the last few seconds, but rather it is understood that the bear is merely getting closer to us. This is called optical expansion. In VR if we want to show an object moving closer to the user then that object's scale should increase. 


In the fine arts, it is common to use the principle of linear perspective to create depth. This is done by creating lines that converge to a single point, this creates the illusion of depth and distance. 
\begin{figure}[!ht]
	\begin{center}
		\woopic{shadows}{.6}
	\end{center}
	\caption{Placement of objects can be understood based on their shadow's position \cite{LaValle2017}.} \label{fig:shadows}
\end{figure}Figure~\ref{fig:Perspective} shows how linear perspective can create the illusion of depth. 


In the fine arts it is also common to use shadowing. The placement of shadows can show the placement of objects, if there is a shadow touching the object we can assume it is on the ground. If the shadow is under the object then the object appears to be in the air and so on. Shadows can also indicate the relative size and shape of objects. The use of shadows to create a different perception of an object's placement is shown in Figure~\ref{fig:shadows}. 


Lastly, objects that are far in the distance have less visible detail than objects that are closer. For example, when holding a rose it is possible to see each petal, thorn, and leaf. When that rose is at a distance, at the end of a football field for example, it is likely that those details will be lost and the general shape and color is all that remains. 


These tactics can be used to trick the brain into thinking that the world is 3D. The sense of depth is vital in convincing the user that they are immersed in a real world. Occlusion, motion parallax, relative size, optical expansion, linear perspective, shadows, and use of detail can all be used in a virtual environment to create a more believable world. 


\subsubsection{Seeing Color}
Colors can be perceived to have three main components, hue, saturation, and value. This classification of color is the color space called HSV (hue, saturation, value). HSV is a color space that is easier to understand than the commonly used RGB (red, green, blue) system. Section~\ref{DisplayChap} discusses how a computer interprets RGB values and how it displays colors. In the HSV system, hue is the actual color, for example red or yellow. Saturation is the purity of the color, how vibrant or pastel it is. Value corresponds to the brightness. This is shown in Figure~\ref{fig:colors}.  
\begin{figure}[!ht]
	\begin{center}
		\woopic{colors}{.6}
	\end{center}
	\caption{A representation of the color wheel with hue, saturation, and value \cite{LaValle2017}.} \label{fig:colors}
\end{figure} To create a realistic virtual world the displayed color of objects should change when in different situations, lighting, surrounding, placement, etc. It is understood that if someone is standing within a shadow, the part of their shirt that is in the shadow is the same color as the part of their shirt that is in the sun. This is known as color consistency and should be applied in the VR world. 

\subsection{Hearing in Virtual Reality}
The sense of sound is an integral part of the the human experience. While developing for VR it is important to understand how the human ear processes sound and how to recreate realistic sounds for the user. Sound can be used to imply movement, direction, situational awareness, narration, confirmation or feedback, and different actions. The user can hear a sound from anywhere in the 360-degree space, giving them more information about their environment than even sight in some cases \cite{Madole1995}. 

\subsubsection{Sound Intensities}
There are many factors that control the volume of sound; proximity to the listener or type of sound, for example. When developing for VR it is important to keep in mind the volume of each sound in the environment. The ambient sounds of birds should not be louder than a nearby non-player character's (NPC) dialog. Figure~\ref{fig:soundVolume} shows the intensities for different sounds. It is important to use sound to help the player better understand and be immersed in the environment. It is important to keep in mind that the average human begins to feel pain with sounds that exceed 120 dB \cite{JIS2}. 
\begin{figure}[!ht]
	\begin{center}
		\woopic{soundVolume}{1.4}
	\end{center}
	\caption{Common Environmental Noise Levels \cite{JIS2}.} \label{fig:soundVolume}
\end{figure}


Sounds can be used to display information about the environment. For example, when a sound is muffled it can be deduced that it's source is far away. This is called attenuation. The movement of sounds can also be detected, as the pitch of a sound increases it can be perceived that the source is moving towards the listener. This is referred to as the Doppler shift \cite{JIS2}. Sound waves move slower that light waves, because of this, there is a delay between a person being able to see a sounds' creation and being able to hear the created sound. For example, it takes a few moments for someone to hear the thunder from a lightning bolt that struck a few miles away. This is important to keep in mind when developing sound for VR environments in order to keep the user immersed and keep the environment believable \cite{Madole1995}. 


Objects in the environment can also effect sound. To create a realistic outdoor environment factors such as wind and ground cover need to be taken into account. For instance, an environment containing wind sounds different than one without. The position of the user within the wind also matters, whether they are downwind or upwind. An environment surrounded by trees creates different sounds than one in short or thick grasses. The type of material that a sound is bouncing off of should also be taken into account as well as the amount of reverberation or echo a certain material makes. For example a loud sound in a padded room is absorbed much more quickly than a loud sound in an opera house \cite{Madole1995}.   


Sounds should match the action in the environment. For example, if the user throws an item in the trash in VR it should sound like they are doing so and not like a dog bark. To create sounds it is possible to use synthetic or recorded sounds. Sounds can be created by changing the speed, pitch or combining two sounds. Foley sound is the process of creating sounds to fit a certain scene. Sound effect artists use every day objects to create surreal sounds, for example using a frozen turkey being slapped to give a punching sound a denser quality \cite{Madole1995}.   


\subsubsection{The Human Ear}
Humans have the ability to determine the direction of a sound. This can be useful in a virtual environment to help the player perceive their environment. The spherical coordinate system is used to describe location of sounds based on a human origin, as shown in Figure~\ref{fig:audiocoordinates}. 
\begin{figure}[!ht]
	\begin{center}
		\woopic{audiocoordinates}{.4}
	\end{center}
	\caption{Common Environmental Noise Levels \cite{JIS2}.} \label{fig:audiocoordinates}
\end{figure}The "common spherical coordinate system [is] based on three elements: azimuth (left or right of straight ahead on the horizontal plane, with 0 degrees azimuth directly in front, 90 degrees to the right, and 180 degrees  directly behind), elevation (the angle above or below the horizontal plane), and distance" \cite{JIS2}. In VR, sounds can be placed on this coordinate system to give the user a sense of space and direction. 
%Expand

\subsection{Feeling in Virtual Reality}
There are two main kinds of haptic sensation, kinesthetic (force) feedback and tactile feedback. Kinesthetic feedback is sensed by the muscles, joints and tendons while tactile feedback is sensed through the skin. With today's technology tactile feedback is much more commonly used as opposed to kinesthetic feedback.

 
Tactile feedback can be extremely beneficial to have in VR environments. It can create a more immersive environment, allow for easier navigation in the world, and tell the player that they have successfully completed a task. This kind of tactile feedback is known as haptics. An example of haptic feedback are the vibrations in a controller as the player presses a button to open a door. In reality there are many ways to explore an object through touch. It is possible to find the size, shape, weight, heat, firmness, and surface texture of an object by handling it. Figure~\ref{fig:hapticExplortion} demonstrates these interactions \cite{LaValle2017}. 
\begin{figure}[!ht]
	\begin{center}
		\woopic{hapticExploration}{.6}
	\end{center}
	\caption{Interactions with objects \cite{LaValle2017}.} 	\label{fig:hapticExplortion}
\end{figure}VR is still limited in the use of all of these tactile interactions. In most VR systems there is a controller for the user to interact with the virtual environment. That controller can vibrate to give the user haptic feedback but that is the extent of its feedback ability. Vibrations are not able to create kinesthetic feedback. A controller can not give the user information about an objects hardness, texture, temperature, weight, or shape. There are new technologies however that are being developed that increase the amount of information users can gain from haptic feedback. These technologies usually consist of a glove that houses other gadgets that can manipulate the hand to create the sensation of kinesthetic feedback \cite{Mazuryk}. Figure~\ref{fig:forceFeedback} shows an example of one of these kinds of gloves. 
\begin{figure}[!ht]
	\begin{center}
		\woopic{forceFeedback}{.6}
	\end{center}
	\caption{Glove that can create sensations of kinesthetic feedback \cite{Mazuryk}.} \label{fig:forceFeedback}
\end{figure}A newer device being produced is called the Gloveone, it is a lightweight glove that contains thousands of sensors that can be activated to vibrate through interaction with virtual objects. The glove's sensors can create 1024 intensities, therefore thousands of combinations. It also tracks the movement and placement of the hand on the plane (x,y,z). The glove has great potential and is continuing to be developed \cite{JIS2}. 


Haptic feedback can create a link between the user and the VR world. The rubber hand illusion illustrates this idea well. In this illusion, the participant sits at a table with both arms outstretched and resting in front of them. The participants' left hand is covered with a cloth and a rubber hand is placed in its spot. The experimenter then stokes both the rubber hand and the participant's hand with a brush, this builds visual and touch association with the fake forearm. The experimenters found that the same part of the participant's brain was activated even when interacts with the fake hand. After this connection was made the experimenter made a stabbing gesture with a needle at the rubber hand. The participant reacted with an anticipation of pain and tended to pull back their real hand, although the hand was never actually threatened \cite{LaValle2017}. 
\begin{figure}[!ht]
	\begin{center}
		\woopic{RubberHand}{.8}
	\end{center}
	\caption{Rubber hand experiment \cite{LaValle2017}.} \label{fig:RubberHand}
\end{figure}They also found that hot and cold sensations could be perceived by association as well. This is called the body transfer illusion. This concept can be applied to VR and haptics. If the player can achieve this body transfer illusion, immersion is more successful \cite{LaValle2017}. 

\subsection{Side Effects}
Side effects of using VR have been found to include nausea, eye strain, disorientation, and other forms of discomfort. Some studies show 80 to 90 percent of people using head-mounted displays felt one or more of these side effects \cite{JIS2}. Visually induced motion sickness (VIMS) has also become an issue with players of fully immersive VR systems. VIMS includes symptoms such as nausea, increased sweating, drowsiness, disorientation, dizziness, headaches, difficulty focusing, blurred vision, and occasionally vomiting \cite{JIS2}. These symptoms are known to vary between people but can begin within minutes of entering a VR experience. Symptoms can also continue for hours after exiting the VR environment. There are many theories as to why VIMS happens to so many people, Figure~\ref{fig:sickness} shows the many causes for each symptom. However there are definite technological factors that contribute to these symptoms.  
\begin{figure}[!ht]
	\begin{center}
		\woopic{sickness}{.6}
	\end{center}
	\caption{Causes of Side Effects can not always be identified \cite{LaValle2017}.} \label{fig:sickness}
\end{figure}Latency is one of these technological culprits. Latency is when there is a time gap between the motion of the player and the system's response. Although VIMS is persistent, some users have been able to adapt to a virtual environment with repeated exposures. As discussed in subsection~\ref{sec:Binocular Cues}, the screen of a head-mounted display sits inches from the users eyes. This causes the eyes to constantly be in a state of convergence which leads to immense eyestrain as it is not a natural position for the eye. Another factor that can contribute to sickness is the spacing between the screens; this distance needs to be determined based on the distance of the users eyes and varies from person to person. As a better understanding of the human sensory system is accomplished, perhaps negative side effects such as VIMS will no longer be an issue.


\section{Hardware}
There are many kinds of virtual reality hardware. Today most companies creating displays for VR use head mounted displays. Head mounted displays (HMD) are devices worn on the head that usually house two small optic displays, one in front of each eye. Some HMD's require a computer to run, the Oculus Rift and the HTC Vive for example. Other HMD's use gaming consoles to run such as the PlayStation VR. Finally some HMD's use phones, such as Android or IOS, to run the display. These are usually the more inexpensive option. Phone driven HMD's include the Google Cardboard, Samsung Gear VR, and Google Daydream View, all of these devices are headsets that use a phone, which is placed inside the headset, to project the image. This study uses the Oculus Rift. 


The hardware for the Oculus Rift can be categorized into three main groups; input, a computer, and output. The input consists of information extracted from the real world, for example a pressed button or a head movement. A computer takes that input and processes it to create the proper output. The output corresponds to a stimulus for a sense organ, such as a display or a sound \cite{LaValle2017}. In the Oculus, information moves from the headset to the computer by way of a 10 foot cable \cite{Sharkey2012}.


\subsection{Display} \label{DisplayChap}
A benefit to using head mounted displays is that the user can complete a 360-degree view of the virtual world as they move around. The field of view (FOV) however is usually in the range of 80-100 degrees horizontal. This limited FOV has been found to hinder the users ability to perceive the world as images may become distorted with perception of size and distance being affected. 
\begin{figure}[!ht]
	\begin{center}
		\woopic{insidertheRift}{1}
	\end{center}
	\caption{Inside the Rift \cite{RajeshDesai2014}.} \label{fig:insidetheRift}
\end{figure} 


To provide a 360 degree world with a FOV of 80-100 degrees the user must be able to look around. In order to accomplish this, the Oculus Rift needs to track the users movements and move the displayed image accordingly. This can be done by the Oculus identifying the placement of the HMD. The Oculus Rift uses an inertial measurement unit (IMU) to find the orientation of the headset. To do this the IMU uses gyroscope, accelerometer, and magnetometer sensors \cite{LaValle2017}. The IMU is located on the Oculus Rift's circuit board, which is shown in Figure~\ref{fig:insidetheRift}. The gyroscope measures angular velocity along the three orthogonal axes, accelerometers measure linear acceleration along three axes, and magnetometers measure magnetic field strength along the three perpendicular axes \cite{LaValle2017}. The lenses are placed so that the screen appears to be "infinitely far" away to the user and still fills most of their field of view. The lenses can be seen in Figure~\ref{fig:insidetheRift}. This is done in VR headsets by putting the screen very close to the user's eye. This then fills their field of view making it so they do not see the real world. With the screen being so close it is impossible for the human eye focus on it naturally so, convex lenses are used to allow the eyes to focus. Convex lenses act similarly to reading glasses because of their use of magnification. To see a clear picture people may need different distances between them and the scree. The Oculus Rift features a nob that allows the user to change the distance between the screen and the lenses, see Figure~\ref{fig:insidetheRift}. 
\begin{figure}[!ht]
	\begin{center}
		\woopic{RBGcolors}{.7}
	\end{center}
	\caption{Color creation in an RGB triangle \cite{LaValle2017}.} \label{fig:RGBcolors}
\end{figure} This allows individuals to change the distance between the lenses and the screen, allowing them to find their clearest picture. There is also a nob that allows the user to change the distance of the lenses from each other to match up with their own facial structure. 


The Oculus Rift uses Organic Light Emitting Diode (OLED) technology for its display, which has a resolution of 960 x 1080 \cite{Sharkey2012}. In OLED displays each pixel has it's own light source, so pixels are not back lit, allowing for darker blacks in comparison to other display systems. The OLED display has self-emitting organic compounds for red, green, and blue. Each color has a value from 0 (dark) to 255 (bright), 0 implies the light is off and 255 implies the light is at full power. Red can be made with the RGB value (255, 0, 0), green is (0, 255, 0) and blue is (0, 0, 255). These colors can be combined to create any visible color, this concept is called trichromatic theory \cite{LaValle2017}. When the RGB values are all equal, (255, 255, 255), then white light is created. Black can be created with RGB values of (0, 0, 0). Figure~\ref{fig:RGBcolors} shows how red, green, and blue can be combined to create all colors. 



%HOW TRACKING WORKS
%\begin{figure}[!ht]
%	\begin{center}
%		\woopic{riftOrientation}{.8}
%	\end{center}
%	\caption{Right-hand rule of rift tracking \cite{Sharkey2012}.} \label{fig:riftOrentation}
%\end{figure}


\subsection{Audio}
There are many levels of sophistication when it comes to audio quality, 3D stereo sound is one of the premiere options. 3D stereo sound takes into account the location of the user in the environment and delivers sound accordingly. This adds to the immersion of the game making it feel more realistic and life-like \cite{JIS3}. It is common practice to use regular over ear headphones to generate the audio for the environment. 


\subsection{Controllers}
The Oculus uses a controller called the Oculus Touch, shown in Figure~\ref{fig:controller}. It has the ability to vibrate at different intensities and for different lengths of time. It also has a multitude of input buttons and is an easy shape to grip. 
\begin{figure}[!ht]
	\begin{center}
		\woopic{controller}{1.9}
	\end{center}
	\caption{Oculus Touch controller from Unity Documentation.} \label{fig:controller}
\end{figure} The Oculus Touch has touch sensors that can detect one or more objects (fingers) being in contact with or hovering over the sensor, this allows for the user to use the controller without needed to apply much force \cite{Publications2018}. 


\section{Conclusion}
Virtual Reality is an entirely manufactured experience. Designers of VR worlds must take into account every sight, sound, and haptic feedback the user encounters with the goal of creating a believable world. To create a realistic feeling world, it must be as close to reality as possible. To do this is it important to understand how the human body works and how to stimulate the human senses. 




%!TEX root = ../username.tex
\chapter{Software}
There are many different ways to develop for VR. Unity3D and Unreal Engine are two free game engines that support VR development. Both are very popular and have their own pros and cons. Unity3D uses its own language called C\# while Unreal uses C++. This study uses Unity3D. Within every game engine the basic principles specifically how objects are created, transformed, and then displayed.  

\section{The Virtual World}
In the virtual world, each object is given a set of coordinates based on the right-handed coordinate system, shown in figure~\ref{fig:objectPlacement}. 
\begin{figure}[!ht]
	\begin{center}
		\woopic{objectPlacement}{.7}
	\end{center}
	\caption{Objects receive a set of coordinates based on the right-handed coordinate system 
		\cite{LaValle2017}.} \label{fig:objectPlacement}
\end{figure}The up axis can be specified based on preference. One common practice uses y = 0 to correspond to the floor or sea level of as environment. The location of x = 0 and z = 0 are then in the center of the virtual world, which divides it into quadrants based on sign \cite{LaValle2017}. 


To create a 3D object, a mesh of that object is created. A mesh is an arrangement of numerous triangles in a way that the final object can be identified as its intended form. The amount of triangles in the mesh can determine the detail of the object. A mesh with fewer triangles creates a less detailed geometric figure while a mesh with more triangles can create a more detailed and realistic form. There are two mesh types, coherent and non-coherent models. If the triangles that create the mesh are oriented so that the they "fit together perfectly so that every edge borders exactly two triangles and no triangles intersect unless they are adjacent along the surface" then they are a coherent model \cite{LaValle2017}. This is because there is a clear inside and outside of the model. If we were to fill the model with gas, none of the gas would leak out. A model that is not closed in this way is a non-coherent model and is not easy to work with \cite{LaValle2017}. Figure~\ref{fig:meshDolphin} is an example of a coherent model.
\begin{figure}[!ht]
	\begin{center}
		\woopic{meshDolphin}{.8}
	\end{center}
	\caption{Mesh for a dolphin \cite{LaValle2017}.} \label{fig:meshDolphin}
\end{figure}


Triangles are used to create an object's mesh because they are convex primitives, lying in one plane. Algorithms that  handle triangles require the least computational cost than other. Other primitives like squares or rectangles are used but are translated into component triangles to avoid computational cost. It is important to have the least amount of computational cost so the game can run smoothly. Models can either be stationary or movable within the virtual world. Stationary models keep the same coordinates forever, like streets or buildings. Movable models can be moved or transformed into varying sizes, positions and orientations. Movable models can be objects such as cars or avatars \cite{LaValle2017}. Models can have many reasons to move such as physics, an environmental factor or because of a user's interactions \cite{LaValle2017}. If there is an interaction between two objects or the user and an object forces and other properties need to be taken into account, such as speed, mass, and gravity. The physics based movement of objects is discussed further in section~\ref{physics}.   


To move an object in the virtual world while keeping it's original size and shape consistent, values (x, y, z) are added to each of the mesh's triangle's coordinates. For example if we have the 3D triangle,
\[((x_{1}, y_{1}, z_{1}), (x_{2}, y_{2}, z_{2}), (x_{3}, y_{3}, z_{3}))\]
and we move it by the amounts $x_{t}$, $y_{t}$, and $z_{t}$ then the new coordinates are;
\[(x_{1}, y_{1}, z_{1})  \rightarrow ((x_{1} + x_{t}), (y_{1} + y_{t}), (z_{1} + z_{t}))\] 
\[(x_{2}, y_{2}, z_{2})  \rightarrow ((x_{2} + x_{t}), (y_{2} + y_{t}), (z_{2} + z_{t}))\] 
\[(x_{3}, y_{3}, z_{3})  \rightarrow ((x_{3} + x_{t}), (y_{3} + y_{t}), (z_{3} + z_{t}))\]


To change the scale of an object, a constant (x, y, z) is multiplied to each of the mesh's triangle coordinates. If we observe the same triangle as before the new coordinates are as follows;
\[(x_{1}, y_{1}, z_{1})  \rightarrow ((x_{1} * x_{t}), (y_{1} * y_{t}), (z_{1} * z_{t}))\] 
\[(x_{2}, y_{2}, z_{2})  \rightarrow ((x_{2} * x_{t}), (y_{2} * y_{t}), (z_{2} * z_{t}))\] 
\[(x_{3}, y_{3}, z_{3})  \rightarrow ((x_{3} * x_{t}), (y_{3} * y_{t}), (z_{3} * z_{t}))\]


For a model's orientation to change, it needs to be rotated. There are three degrees of rotational freedom in 3D environments. These are shown in Figure~\ref{fig:rotaion}. 
\begin{figure}[!ht]
	\begin{center}
		\woopic{rotation}{.8}
	\end{center}
	\caption{Three-dimensional rotations \cite{LaValle2017}.} \label{fig:rotaion}
\end{figure}The yaw, pitch, and roll rotations can be combined to rotate a model in any direction. To rotate an object about one of the x, y, or z axis, each coordinate in the mesh's triangle's is multiplied by a matrix. The following is the general form of the matrices to calculate the rotation of an object around the z axis. The vector B contains the original (x, y, z) coordinates of the coordinate in the mesh's triangle's. The Greek letter $\gamma$ in the 3 X 3 matrix is the angle of rotation. The values in the third row of the 3 X 3 matrix allows for the z coordinates to be unchanged, as they should not change when rotating around the z axis \cite{LaValle2017}.
\[
\begin{bmatrix}
	A_{x} \\
	A_{y} \\
	A_{z} 
\end{bmatrix}
=
\begin{bmatrix}
	cos \gamma & -sin \gamma & 0  \\
	sin \gamma & -cos \gamma & 0  \\
	0 & 0 & 1 \\
\end{bmatrix}
\begin{bmatrix}
	B_{x} \\
	B_{y} \\
	B_{z} 
\end{bmatrix}
\]
To change the rotation of the objects on the x or y axis the following matrices are used. Notice that the items in rows that correspond to the axis being rotated around, are not changed from the identity matrix;
\[
R_{x}(\beta) = 
\begin{bmatrix}
1 & 0 & 0 \\
cos \beta & -sin \beta & 0  \\
sin \beta & -cos \beta & 0  \\
\end{bmatrix}
\]
\[
R_{y}(\alpha) = 
\begin{bmatrix}
cos \alpha &  0 & sin \alpha  \\
0 & 1 & 0 \\
-sin \alpha & 0 & cos \alpha \\
\end{bmatrix}
\]


There are two main ways objects can be rendered in virtual reality, both methods require a nested loop. The first is called object-order rendering. This works by analyzing which triangles are in the users field of view, the triangles that are closest to the eye will have their pixels updated first. For this method, the outer loop iterates over the triangles and the inner loop iterates over the pixels. Image-order rendering uses the outer loop to iterate over the pixels and the inner loop to find which triangles influence the RGB of that pixel \cite{LaValle2017}. 
\begin{figure}[!ht]
	\begin{center}
		\woopic{raycasting}{.7}
	\end{center}
	\caption{Determining the first intersection between the users line of sight and the objects in the environment, this is done to find what triangles should be rendered \cite{LaValle2017}.} \label{fig:raycasting}
\end{figure} There are two steps for rendering a pixel; ray casting (finding what to render) and shading (deciding what RGB it should have). Ray casting involves creating a ray from the eye into the virtual world. The nearest point of intersection among the triangles in the world will be the one that gets rendered. Shading is the calculation of the RGB of that pixel based on lighting conditions and material properties. Object-order rendering is much more efficient than image-order rendering because of this process \cite{LaValle2017}. To add texture to an object it is possible to add a repeating pattern to the surface of the object to give the illusion of texture, this is called texture mapping. 


To improve performance and a faster frame rate a practice called culling is often used. Culling is a preprocessing phase of the rendering process. It eliminates some of the triangles from a scene that need to be rendered. One form of culling is called view volume culling. View volume culling is where the triangles that are outside the users field of view are eliminated. Occlusion culling is the elimination of triangles that would be hidden from view from a closer triangle. Another way to improve performance is to reduce the amount of triangles in the models, this may result in a loss of detail in the model but increase performance \cite{LaValle2017}. 


\subsection{Physics}\label{physics}

\subsection{Audio}
For an effective sound display there are three categories that must be met; sound generation, spacial propagation, and mapping of parameters. Sound generation is the playback or creation of sound that follows an action, steps or object collisions for example. Spacial propagation is computationally expensive and involves calculating how sound wave would move in the environment, echoes off of objects of different shapes and materials for example. Finally mapping of parameters implies the the proper sounds being played in relation to the calculated parameters \cite{Mazuryk}. 


\section{Introduction to Unity}
Unity is a free to download game engine. Unity uses scripts, written in C\#, that can be attached to objects allowing that code to be applied to said object. It has a large set of assets, objects, available for free at the unity Asset Store. 


\subsection{Locomotion}

\subsection{User Input}



\section{Conclusion}





%\input{chapters/chapter6}
%\input{chapters/chapter7}
%!TEX root = ../username.tex
\chapter{Conclusion}

\section{Game Results}
Feedback - Survey to see if UI is good and if people learned anything. 

\section{Future Work}
Game expansion, Lab work, more story line, evolve to modern human




%%%%%%%%%%%%%%%%%%%%%%%%%%%%%%%%%%%%%%%%%%%%%%%%%%%%%%%
%
%  This section starts the back matter. The back matter includes appendices, indicies, and the
%  bibliography
%
%%%%%%%%%%%%%%%%%%%%%%%%%%%%%%%%%%%%%%%%%%%%%%%%%%%%%%%

\backmatter

%\input{appendices/math}
%\input{appendices/java}
%\input{appendices/cpp}
%!TEX root = ../username.tex
\chapter*{Afterword}\label{after}
\addcontentsline{toc}{chapter}{Afterword}
\markboth{Afterword}{Afterword}




%%%%%%%%%%%%%%%%%%%%%%%%%%%%%%%%%%%%%%%%%%%%%%%%%%%%%%%
%
%  This section would be used if you are not using BibTeX. Look at Kopka and Daly for how to
%  format a bibliography manually as well as how to use BibTeX.
%
%%%%%%%%%%%%%%%%%%%%%%%%%%%%%%%%%%%%%%%%%%%%%%%%%%%%%%%

%\begin{thebibliography}{99}
%\bibitem{}
%\bibitem{}
%\end{thebibliography}

%%%%%%%%%%%%%%%%%%%%%%%%%%%%%%%%%%%%%%%%%%%%%%%%%%%%%%%
%
%  We used BibTeX to generate a Bibliography. I would recommend this method. However, it is
%  not required.
%
%%%%%%%%%%%%%%%%%%%%%%%%%%%%%%%%%%%%%%%%%%%%%%%%%%%%%%%

\renewcommand\bibname{References} % changes the name of the Bibliography

\nocite{*} % This command forces all the bibliography references to be printed -- not just 
              % those that were explicitly cited in the text.  If you comment this out, the bibliography
              % will only include cited references.
\ifthenelse{\boolean{woosterchicago}}{
\bibliographystyle{woosterchicago}}{\ifthenelse{\boolean{achemso}}{
\bibliographystyle{achemso}}{\bibliographystyle{plainnat}}}
% if you have used the woosterchicago class option then your references and citations will be in Chicago format. If you have used the achemso class option then your references and citations will be in the American Chemical Society format. If you do not specify a citation format then the default Wooster format will be used.
\bibliography{references} % load our Bibliography file

%%%%%%%%%%%%%%%%%%%%%%%%%%%%%%%%%%%%%%%%%%%%%%%%%%%%%%%
%
%                                                                Index
%
%  Uncomment the lines below to include an index. To get an index you must put 
%  \index{index text} after any words that you want to appear in the index.
%  Subentries are entered as \index{index text!subentry text}. You must also run the
%  makeindex program to generate the index files that LaTeX uses. The PCs are set to run
%  makeindex automatically.
%
%%%%%%%%%%%%%%%%%%%%%%%%%%%%%%%%%%%%%%%%%%%%%%%%%%%%%%%

\ifthenelse{\boolean{index}}{
\cleardoublepage
\phantomsection
\addcontentsline{toc}{chapter}{Index}
\printindex}{}

%%%%%%%%%%%%%%%%%%%%%%%%%%%%%%%%%%%%%%%%%%%%%%%%%%%%%%%
%
%                                                                Colophon
%
%  A Colophon is a section of a printed document that acknowledges the designers and printers of the work.
% The colophon also includes information about the fonts and paper used in the printing. It is not required 
% for your IS and can be commented out.
%
%%%%%%%%%%%%%%%%%%%%%%%%%%%%%%%%%%%%%%%%%%%%%%%%%%%%%%%

\ifthenelse{\boolean{colophon}}{
\begin{colophon}
This Independent Study was designed by Dr. Jon Breitenbucher.\newline
It was edited and set into type in Wooster, Ohio,\newline
using the \ifthenelse{\boolean{xetex}}{\XeTeX\ typesetting system designed by Jonathan Kew}{\LaTeX\ typesetting system designed by Leslie Lamport}\newline
and based on the original \TeX\ system of Donald Knuth.\newline
It was printed and bound by Office Services at The College of Wooster.

The text face is Adobe Garamond Pro, designed by Robert Slimbach.\newline
This is the Opentype version distributed by Adobe Systems\newline
and purchased as part of the Adobe Type Classics for Learning.

The paper is standard laser copier paper and not of archival quality.
\end{colophon}}{}
\clearpage\thispagestyle{empty}\null\clearpage
\end{document}